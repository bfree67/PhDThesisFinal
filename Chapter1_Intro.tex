\chapter{Introduction}

Poor ambient air quality was estimated to cause over 9 million deaths worldwide in 2015 and up to 25\% of all deaths in some severely affected countries \citep{Landrigan2017}. In addition to direct health impacts and enhanced mortality, air pollution is also estimated to cost world economies from \$4-6 trillion USD annually due to lost time and health care costs \citep{Landrigan2017, Narain2016}, with over \$1.57 Trillion USD lost in Europe alone \citep{Roy2015}.

Currently, over 92\% of the world's population live in areas where air quality exceeds World Health Organization (WHO) limits \citep{WHO2016} with many of these people living in dense, urban centers. Air pollutants include primary products of combustion, products of incomplete combustion, fugitive emissions, biogenic and agricultural sources, and secondary products of combustion caused by photochemical transformation. The US Environmental Protection Agency (USEPA) lists 188 hazardous air pollutants (HAPs) that generators are required to report on an annual basis \citep{USEPA1996}. The primary pollutants that most regulatory agencies monitor as part of their ambient air quality program is a subset of that list that includes sulfur dioxide (SO$_{2}$), carbon monoxide (CO), nitrogen dioxide (NO$_{2}$), tropospheric ozone (O$_{3}$),  particulate matter less than 10 microns (PM$_{10}$), and particulate matter less than 2.5 microns (PM$_{2.5}$). Each pollutant is generated in part, as a primary or secondary process, from the thermodynamic combustion of fossil fuels.

For developing nations, heavy reliance is placed on highly polluting fuel-based combustion sources to generate electricity, power industry, transport people and products, build infrastructure, and improve the quality of life of its citizens. For many countries, development investment comes from the exploitation of natural resources such as hydrocarbon reserves, minerals, or agricultural products - all industries that are large consumers of fossil fuels.

Regulatory agencies in developing countries often face shortages in skilled personnel, funding and political will in implementing and enforcing environmental regulations and programs to protect air quality \citep{Freeman2015a}. Providing effective baseline studies and tools to establish air management programs are required that can prioritize the limited resources an agency has. 

While installing and maintaining air monitoring stations is a major expense for air management departments,  little effort is given to using the data as a tool to forecast air quality levels and focus emission reduction efforts. Along with knowing what the current and historical air quality concentrations are, the most important information an air manager can have is to know where major polluting sources are, what pollutants they generate, how much they generate of each pollutant, and when will the emissions generated by these sources create harmful concentrations.

The purpose of this research is to create novel air quality management (AQM) approaches that can support initiatives in developing nations. Most countries have air monitoring stations that provide ambient air quality concentrations as well as meteorological parameters. Using historical AMS data sets with open source databases available from various government agencies, stochastic modeling, and machine learning techniques, local air managers can access advanced capabilities that would typically require expensive studies and data collection efforts.

This research used the AQM system prepared for the State of Kuwait under a United Nations Development Program (UNDP) project. A series of tools were developed using machine learning and statistical models to approximate air quality zones, prepare initial national emissions inventory, and forecast air quality concentrations.  These tools were validated with local historical air quality monitoring data along with open access statistics from government sources.

\section{Problem description}

Developing nations face many challenges when establishing a comprehensive air management program. In the rush to appease citizen concerns and political pressures, air managers take shortcuts to achieve results. Air monitoring stations are installed without consideration of hotspot analyses. Standards are adopted that don't consider enforcement procedures. Emission inventories overlook key sectors due to difficulties in quantifying. This lack of key procedures usually results in the ambiguous regulations, confused stakeholders, inconsistent enforcement, and ineffective programs.

Practical understanding of real-world needs was identified during the development of an air management program for the State of Kuwait under the UNDP Kuwait Integrated Environmental Management Systems (KIEMS) project \citep{UNDP2012}.  The main challenges identified during the KIEMS project can be decomposed into five main research topics that are tangentially related and build off each other. The main questions that this thesis asks are:

\begin{itemize}
\item{How can different airsheds and zones be differentiated within an area?}
\item{How can air quality standards be applied to air quality zones if there are multiple monitoring stations and criteria?}
\item{How can small dispersed area sources be included into emission inventories?}
\item{How can parameters for mobile source models be collected?}
\item{How can air pollutant concentrations be predicted based on historical and prevailing ambient conditions?}
\end{itemize}

The first topic, identification of air quality zone led to a novel way to discriminate between coastal and inland wind patterns. Differentiating wind patterns is important to delineating zone with common features in order to improve local management of air pollution generators.  If one zone constantly has ambient air quality problems, that zone should have a higher priority for resources than other zones that have fewer generators or better air quality.

Once air zones are identified, it is important to determine if the zone meets air quality standards. This can be a difficult question if there are more than one air monitoring stations in the zone. Does the air department take the readings from the first station that measurements that exceed the standard or should they be aggregated? The second topic of this research presents a method that incorporates multiple station readings to determine the attainment status of an air zone.  

Emission inventories are important registers that record individual sources of pollution and emissions. A detailed Tier III emissions inventory can identify specific processes and associated pollutants that allow air managers to accurately target sources. The same sources identified for HAPs are usually the same sources emitting GHGs - allowing a synergy of inventory to take place in order to avoid duplication of work. Some sources are more difficult to quantify than others due to large numbers, such as mobile sources, or their distributed nature, such as area sources. The third task presents a novel approach to quantifying small, dispersed sources using MCA and the Central Limit Theorem. These small sources are often overlooked in emission inventories but can contribute significantly when aggregated.

Mobile source emissions play another major component in emission inventories but are usually incorporated through a complex model. Mobile source models require input parameters that should reflect local conditions. The fourth topic provides a novel way of estimating key inputs such as vehicle densities on roads and fleet compositions using a commercial drone.

The last topic, predicting air quality exceedances, allows for short-term forecasts of air quality conditions within the zones using data collected from air monitoring stations. A novel method using deep learning and historical data is presented, as well as procedures to reduce inputs and pre-process training data. 

The relationship of each task to each other within the overall framework of air quality management is shown in Figure \ref{fig:tasks}.
%
\begin{figure}[H]
\centering
\includegraphics[width=.75\textwidth]{images/tasks}  %assumes jpg extension
\caption{Relationship of research tasks.}
\label{fig:tasks}
\end{figure}

The tasks are shown as black-bordered ovals while local data inputs are shown as green-bordered rectangles.
%

\section{Research Needs}

Unlike other media,  air pollution is not easily quantified and changes rapidly based on weather conditions. Water quality and soil conditions can be evaluated visually through discoloration, damaged flora, and dead fauna. Soil contamination is often local with limited dispersion and migration. Air pollutants, on the other hand, can be released thousands of miles away and mix with locally produced emissions, or be transformed into secondary products that bare little resemblance to their initial state. Many of the US EPA defined HAPs are gaseous with no color or smell at harmful levels as shown in Table \ref{tb:odor} \citep{Murnane2013}. In order to determine concentration levels, air quality must be sampled and measured with complex instruments over time.

\begin{table}[H]
\centering
\caption[Visual and odor qualities of common air pollutants]{Visual and odor qualities of common air pollutants (Murnane, 2013)}
\label{tb:odor}
\begin{tabular}{@{}ccccc@{}}
\toprule
\textbf{HAP} & \textbf{CAS} & \textbf{Color} & \textbf{Odor} & \textbf{Odor Threshold (ppm)} \\ \midrule
SO$_{2}$ & 7446-09-5 & No & Yes & 0.33 - 8 \\
NO$_{2}$ & 10102-44-0 & Yes & No & N/A \\
NO & 10102-43-9 & No & No & N/A \\
CO & 630-08-0 & No & No & N/A \\
O$_{3}$ & 10028-15-6 & Yes & Yes & 3.1E-4 - 0.25 \\
H$_{2}$S & 7783-06-4 & No & Yes & 4E-5 - 1.4 \\
Benzene & 71-43-2 & No & Yes & 0.47 - 313 \\
NH$_{3}$ & 7664-41-7 & No & Yes & 0.043 - 60.3 \\ \bottomrule
\end{tabular}
\end{table}

Air quality management for an area begins with one point of measurement. That point is usually a sampling tube mounted at around 10 m above the ground and (hopefully) far from busy roads, exhaust vents, and smokestacks. The sample of the ambient air is therefore assumed to be well mixed and represent the prevailing blend of different upwind emissions generators, and not weighted by a nearby source. The sample is split and drawn into sensors that measure the concentration of a specific parameter (or family of parameters) over an averaging measurement period (usually every five minutes). The measurement period is aggregated into larger time weighted averages, such as 1 hr, 3 hrs, 8 hrs, or 24 hrs, for each parameter, depending on local air quality standards. 

If the average measured concentration is below the standard, the air quality is acceptable. If it is above the standard, it is an exceedance. If we accept the reasonable assumption that air quality standards are based on toxicological studies and represent an acceptable balance between allowable pollution releases and human health, ambient air quality concentrations become the measure of performance for air quality.

Just as one sample cannot adequately represent a complex population in statistics, one air monitoring station cannot adequately represent a country with complex topographies, urban centers, industrial activities, transportation networks, weather patterns, and proximity to neighbors with their complex environments.  In short, the air quality in one part of the country will most likely be different from the other parts. For regulatory agencies with limited staff and budget, prioritizing areas with poor air quality is a more efficient use of resources than applying the same enforcement standards uniformly. Identifying areas for enforcement jurisdiction is, therefore, an important basis of an air management program. As shown in the previous chapter, initial zones were usually established using existing political boundaries without consideration of micro-climate effects based on geography and land use.  

\subsection{Area of Study}

Most of the data used in this research were collected in the State of Kuwait from 2012 to 2017. Kuwait is a small country of 17,818 km$^{2}$ located on the northwest corner of the Persian Gulf, between longitudes 46.56$^{o}$ - 48.37$^{o}$ East and latitudes 28.51$^{o}$ - 30.16$^{o}$ North with over 499 km of coastline \citep{CIA2015}. It is bordered by Iraq to the north and Saudi Arabia to the south and west. The country is classified as a desert zone with the highest altitude reaching only 300 meters above sea level.   In 2016, approximately 3.1 million people lived in Kuwait \citep{CSB2017} with over 64\% of its annual Gross Domestic Product (GDP) coming from the production of hydrocarbons \citep{KAMCO2013}.  Other industries in Kuwait include power generation and water desalination using heavy oil and natural gas at five sites, steel making using electrical induction furnaces, and food preparation.  The country has over 7,400 km of paved roads and over 1.8 million cars in service \citep{CSB2014}.  Over 98\% of the population lives within 10 km of the coast and are subject to coastal effect winds, caused by the diurnal differential heating/cooling of the sea and land \citep{Crosman2010, Cuxart2014}.  The land-sea breezes (LSB) shift direction and speed over the course of the day, re-circulating pollution back and forth from land sources.

The Kuwait Environment Public Authority (KEPA) is responsible for monitoring environmental conditions and enforcing compliance with the national environmental law. It operates 15 each fixed site AMS’s located along the coast as shown in Figure \ref{fig1:amskuwait}.  The KEPA stations measure and report 1 hr averaged air pollutants and meteorological conditions. The bulk of the AMS's are located within two air quality zones (Central and Southern Coast) which correspond to areas of high population and industrial centers. Access to monitoring data from additional stations operated by the Kuwait Oil Company (KOC) provides an air monitoring density of approximately one monitoring station per 163,150 people. This density is similar to Vancouver, Canada which has approximately 1 per 160,000 people, as compared to around 1 per 440,000 people in the South Coast of California \citep{Marshall2008}.

%  
\begin{figure}[H]
\centering
\includegraphics[width=.75\textwidth]{images/risk1.png} 
\caption{Location of air monitoring stations in Kuwait.}
\label{fig1:amskuwait}
\end{figure}

\section{Organization}

This thesis is organized in a \textit{manuscript} format according to the University of Guelph 2016-2017 Graduate Calendar ``Thesis Format'' section (\url{https://www.uoguelph.ca/graduatestudies/current-students/preparation-your-thesis}).  The chapters are outlined as follows:

\subsubsection*{Chapter 1 - Introduction}

This chapter serves as an overview of the research covered by the thesis by presenting the problem statement and research needs. It describes the connections made between the different research topics and chapters of this thesis.

\subsubsection*{Chapter 2 - Air zone mapping}

In this topic, we develop a novel procedure to identify common dispersion areas with a designated region or country using the assumption that land-sea breezes (LSBs) provide different circulatory wind patterns than inland wind. As part of the methodology, we only consider dispersion patterns at ground level, and not at different altitudes. By identifying common mixing zones within a larger airshed, regulatory policies can be tailored to address zone-based problems that might not be applicable or required to other zones. Targeted processes allow resources to be prioritized and reduce economic impacts to sources outside of zones that are in non-attainment of air quality standards.

This chapter was originally published as a paper, \textbf{Mapping air quality zones for coastal urban centers}. Freeman, B., Gharabaghi, J.  Th\'e, S. Faisal, M. Abdullah, and A. Al-Aseed (2017). \textit{Journal of the Air and Waste Management Association}, Volume 67, Issue 5, December 2016, Pages 565-581, \url{ DOI: 10.1080/10962247.2016.1265025}.

\subsubsection*{Chapter 3 - Attainment classification}

Once zones are established, methods are required to convert the measured concentration data from air monitoring stations into management action. A single monitoring station might register high readings that exceed the established standard for the parameter. Assuming the sensor is calibrated and operating properly, the question becomes how many exceedances are allowable before direct action is taken to reduce generator emissions - thereby impacting economic production by cutting operations, expending capital to improve processes, or changing transportation habits by limiting private vehicle operation. These exceedances may be only three exceedances per year, as is the case for O$_{3}$ in the USA, or 2.5 million premature deaths, as is the case in India \citep{Landrigan2017}.

In this topic,  a novel method to evaluate the effectiveness of exceedance determination methods, especially if there are multiple air monitoring stations within the zone, was developed. Two classifications methodologies were evaluated, although it is general enough to assess any classification approach. The USEPA's Human Health Risk Assessment (HHRA) approach was used to develop the procedure that also used Monte Carlo Analysis to generate ranges of ambient pollutant concentrations within the exposure period.

This chapter was originally published as a paper, \textbf{Evaluation of air quality zone classification methods based on ambient air concentration exposure}. Freeman, E. McBean, B., Gharabaghi, and J.  Th\'e, (2017). \textit{Journal of the Air and Waste Management Association}, Volume 67, Issue 5, December 2016, Pages 550-564, \url{DOI: 10.1080/10962247.2016.1263585}.

\subsubsection*{Chapter 4 - Dispersed area sources}

A key requirement in air management is to understand where pollutants come from in order to impact the processes that generate them. Pollutant source evaluation requires an emissions inventory of sources that includes rates of emission for individual analytes, physical parameters of the release point such as height, release temperature, release velocity, and area of release, as well as the geospatial location of the emission source. It also helps to include an industrial code of the source to evaluate processes within a sector, and an emission source code to group common emission processes \citep{The2008}.

Collection of emission rate data is often challenging as the rates are time dependent based seasonal, diurnal, weekly and production schedules. Emissions are also impacted by the quality of feedstock (\% of sulfur or ash in the material). Accurate collection of real-time emissions using continuous emission monitoring systems (CEMS) are limited only to the analytes being measured and to large sources that can afford to install and operate these systems. Other techniques use some form of parametric emission factor equation - whether local factors that include operational data from a process historian capturing fuel rates, operating temperatures, and combustion efficiencies, or published emission factors such as USEPA's AP-42 that use an input feedstock \citep{USEPA1995}. While some of the AP-42 factors are rated as high quality (B or better), they are nonetheless outdated (published in 1995) and incomplete (despite over 12,000 published factors).

While most EFs were developed to characterize stationary sources, many EFs also provide values for area sources. The area sources are assumed to be local and associated with a specific organization. In many non-industrial or commercial scenarios, small sources exist that aggregate to a significant annual amount when taken together.  These sources may be spread throughout the country (or zones), such as smoking or grilling.

In this research, we look at small sources by evaluating the impact of emission contributions from $nargyla$ water pipes in restaurants and caf\'es. We assume that there are a set number of establishments and only consider non-personal uses.  Monte Carlo Analysis is used to represent ranges of uncertainty and variations within data sets. Emission factors were selected from the published literature.

This chapter was originally published as a paper,  \textbf{Estimating Annual Air Emissions from $nargyla$ Water Pipes in Caf\'es and Restaurants Using Monte Carlo Analysis}. Freeman, B., Gharabaghi, and J.  Th\'e, (2017). \textit{International Journal of Environmental Science and Technology}. \url{DOI: 10.1007/s13762-018-1662-6}.

\subsubsection*{Chapter 5 - Vehicle density estimation in congested environments}

Mobile sources contribute a significant amount of emissions in most countries but are difficult to quantify due to a large number of variables and parameters required for calculation. Essential parameters for estimating total mobile source emission contributions to the emissions inventory includes identifying the types and quantities of vehicles using road segments at particular times of the day to prepare a reasonable model of activity that can then be used to calculate emissions. Evaluating traffic at an intersection allows for discrete analysis of vehicles regarding fleet composition, as well as providing emissions generated during idling and accelerating from rest.

Using a commercial drone and DEM generating software, 3D models were generated of traffic at signalized intersections. The vehicle type was identified to build a fleet composition model and inter-vehicle stacking gaps were measured to create a model that could estimate the number and types of vehicles on a discrete road segment. These values can be used as inputs to more complex mobile emission models.

This chapter was originally submitted as a paper, \textbf{Vehicle Stacking at Signalized Intersections with Unmanned Aerial Systems}.  Freeman,  B., J. Al Matawah, M. Al Najat, B. Gharabaghi, and J.  Th\'e, (2018). Submitted to \textit{International Journal of Transportation Science and Technology}, In Review.

\subsection*{Chapter 6 - Forecasting air quality conditions}

Air quality management relies extensively on time series data captured at air monitoring stations as the basis of identifying population exposure to airborne pollutants and determining compliance with local ambient air standards. In this research topic, 8 hr averaged surface ozone (O$_{3}$) concentrations were predicted using Deep Learning (DL) consisting of a recurrent neural network (RNN) with long short-term memory (LSTM). Hourly air quality and meteorological data were used to train and forecast values up to 72 hours with low error rates. The LSTM was able to forecast the duration of continuous O$_{3}$ exceedances as well. Prior to training the network, the dataset was reviewed for missing data and outliers. Missing data were imputed using a novel technique that averaged gaps less than eight time steps with incremental steps based on first-order differences of neighboring time periods. Data were then used to train decision trees to evaluate input feature importance over different time prediction horizons. 

This chapter was originally published as a paper, \textbf{Forecasting Air Quality Time Series Using Deep Learning}. Freeman, G. Taylor, B., Gharabaghi, and J.  Th\'e, (2018).  \textit{Journal of the Air and Waste Management Association}, Volume 67, Issue 5, December 2016, Pages 550-564, \url{DOI: 10.1080/10962247.2016.1263585}.

\subsubsection*{Chapter 7 - Conclusions and Recommendations }

This final chapter emphasizes the overall contributions made by this research and provides recommendations for future work. 


%---------------------------------------------------------------
%------------------------End of Chapter----------------------
\bigskip
\begin{center}
END OF CHAPTER
\end{center}