\addcontentsline{toc}{chapter}{Abstract}
\begin{abstract}
\thispagestyle{empty}
 \addtocounter{page}{-1}
%\renewcommand\thepage{\roman{page}}

\bigskip
\begin{center}
\textbf{AIR QUALITY MANAGEMENT FOR COASTAL URBAN CENTERS USING STOCHASTIC AND MACHINE LEARNING TECHNIQUES}
\end{center}

\bigskip

\noindent Brian S. Freeman \hfill Advisory Team:\\
\noindent University of Guelph, 2018 \hfill Bahram Gharabaghi, Jesse Th\'e\\
\bigskip

This research addressed specific air management issues faced by regulatory agencies to allow better oversight given constrained budgets and technical staff. Data sets were collected from air monitoring stations in Kuwait to allow development, testing, and validation of the techniques described in the research. The novel methods developed as a result of this research combined time series data from ambient air monitoring stations with statistical testing, stochastic methods, and machine learning processes.

By identifying the limit of land-sea breeze effect, air zone boundaries could be established to manage emission sources better. Skewness and kurtosis statistics from dispersion models, along with the distance from shorelines, were used to train a support vector machine model to discriminate between coastal and inland dispersion effects with 98\% accuracy.

A methodology was developed to evaluate designated air management zone air quality exceedances using random exposure over three years based on local air quality standards. Uncertainty within exposure variables was managed using Monte Carlo Analysis that allowed an unlimited number of air monitoring stations located within the zone to contribute to the overall evaluation result and allow comparison of different characterization methods.

A model using Monte Carlo Analysis was developed to account for wide variation of individual caf\'es and restaurants where nargyla smoking takes place. One result of the analysis revealed that these distributed area sources significantly impact overall annual emissions for small countries. 

A method to determine vehicle fleet composition and vehicle density on a road was developed that employed an unmanned aerial system to capture images of cars stacked at a signalized intersection. The images were processed using photogrammetry software to create a digital elevation model that allowed measurement of distances and fleet composition.

Lastly, predicting future air pollution concentrations was accomplished using a deep learning technique consisting of a recurrent neural network with long short term memory. Datasets from a single air monitoring station were used to train and test the model. The resulting model was able to predict 8 hr averaged ozone out to 72 hours with a Mean Absolute Error of less than 2 ppb, outperforming traditional methods.

\normalsize
\end{abstract}
