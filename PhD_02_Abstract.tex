\begin{abstract}
\thispagestyle{empty}
 \addtocounter{page}{-1}
%\renewcommand\thepage{\roman{page}}

\bigskip
\begin{center}
\textbf{AIR QUALITY MANAGEMENT FOR COASTAL URBAN CENTERS USING STOCHASTIC AND MACHINE LEARNING TECHNIQUES}
\end{center}

\bigskip

\noindent Brian S. Freeman \hfill Advisory Team:\\
\noindent University of Guelph, 2018 \hfill Bahram Gharabaghi, Jesse Th\'e\\
\bigskip

Low ambient air quality is a leading cause of premature deaths for a large percentage of the world's population. Many of thess at-risk people live and work in urban centers on or near coastlines and in developing nations.  Coastal communities face enhanced air quality problems due to land-sea breezes that re-circulate previously emitted primary and secondary air pollutants into airsheds already burdened by emissions from industrial and mobile sources.  For many nations impacted by concentrated air pollution, regulatory agencies do not have the resources to develop comprehensive air management programs similar to the mature programs in the USA and Europe.  This research addressed specific air management issues faced by regulatory agencies to allow better oversight given constrained budgets and technical staff. Data sets were collected from air monitoring stations in Kuwait to allow development, testing, and validation of the techniques described in the research.\\

The novel methods developed as a result of this research combined time series data from ambient air monitoring stations with statistical testing, stochastic methods, and machine learning processes. The fundamental questions considered were:

\begin{itemize}
\item{How can different airsheds and zones be differentiated within an area?}
\item{How can air quality standards be applied to air quality zones if there are multiple monitoring stations and criteria?}
\item{How can small dispersed area sources be included into emission inventories?}
\item{How can parameters for mobile source models be collected?}
\item{How can air pollutant concentrations be predicted based on historical and prevailing ambient conditions?}
\end{itemize}

The first question was addressed by modeling virtual sources throughout a region using an advanced Lagrangian puff air dispersion model and prognostic meteorological data.  Evaluating the distribution of 1-year concentration averages showed that relationships between the distributions’ skewness and kurtosis statistics could predict whether the dispersion patterns were influenced by land-sea breezes or the prevailing inland winds. By identifying the limit of land-sea breeze effect, air zone boundaries could be established to manage individual sources better.  The resulting skewness and kurtosis statistics, along with the distance from the shore, were used to train a support vector machine model, which could discriminate between coastal and inland dispersion effects with 98\% accuracy. \\

The second question regarding the evaluation of a designated air management zone in regards to limits exceedance was addressed using random exposure over three years based on local exceedance standards.  Uncertainty within exposure variables was managed using Monte Carlo Analysis that allocated an unlimited number of air monitoring stations located within the zone to contribute to the overall evaluation result and allow comparison of different characterization methods.  The Central Limit Theorem was used to simplify calculations.\\

Small dispersed area sources include smoking, BBQs, and open burning. In the Middle East,\textit{ nargyla}, or hookah smoking is common.  A model using Monte Carlo Analysis was developed that also employed the Central Limit Theorem to account for wide variation of individual caf\'es and restaurants where \textit{nargyla} smoking takes place.  One result of the analysis revealed that not including these small individual sources significantly impacts overall annual emissions for small countries. In Kuwait, $nargyla$ smoking was calculated to represent over 20\% of all nitrous oxide reported in 1995 based on Kuwait’s Initial Communication to the United Nations Framework for Climate Change Convention. \\

Estimating mobile source emissions is a significant challenge for air managers. Many models exist to calculate annual amounts, but each requires local inputs based on the number and types of vehicles on the road. Collecting these values is a time consuming and expensive process that must be updated annually to account for different road segment use.  A method to determine vehicle fleet composition and vehicle density on a segment of road was developed that employed an unmanned aerial system (drone) to capture images of cars stacked at a signalized intersection.  The collected images were processed using photogrammetry software to create a digital elevation model that allowed measurement of stacking distances and identification of individual vehicles.  Both fleet composition and stacking distances did not vary significantly (p$>$0.05).  The stacking distance followed a log-normal distribution and was transformed before significance testing.  The data were assumed to have equal variance and could be pooled.\\

Lastly, predicting future air pollution concentrations was accomplished using a deep learning technique consisting of a recurrent neural network with long short term memory.  Datasets from a single air monitoring station were used to train and test the model. Prior to training, the data was cleaned using a novel first-order imputation technology to estimate gaps and outliers.  A decision tree method was used to estimate the importance of individual features while a scaling method limited input values to between 0 and 1.  The resulting model was able to predict 8-hour averaged ozone out to 72 hours with a Mean Absolute Error of less than 2 ppb, outperforming traditional methods using multi-layer feed forward networks and ARIMA.\\

As a result of these novel techniques, key requirements of an air quality management program can be rapidly and economically implemented.  This research recommends further evaluation of each method with real-world scenarios to validate their effectiveness further. 

\normalsize
\end{abstract}
