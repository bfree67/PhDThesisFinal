\chapter{Forecasting air quality conditions}

\section{Introduction}

Tropospheric ozone (O$_{3}$) is a secondary pollutant formed by complex photo-chemical processes that impacts human health, plants, and structural materials. More than 21,000 premature deaths in Europe are attributed annually to O$_{3}$ exposure \citep{WHO2008} with over 1.1 million deaths worldwide and accounting for over 20\% of all deaths attributed to respiratory diseases \citep{Malley2017}. The majority of tropospheric O$_{3}$ is generated through anthropogenic sources \citep{Lelieveld2000, Cooper2006} attributed to the photo-disassociation of NO$_{2}$ as shown below in the simplified reaction \citep{Finlayson1993}:

\begin{equation}
\label{eq:ozoneformation}
\begin{gathered}
NO_{2}+h\nu (\lambda < 430nm) \rightarrow NO+O \\
O+O_{2}\overset{M}{\rightarrow} O_{3}
\end{gathered}
\end{equation}

\noindent
where $M$ is a stabilization molecule used during the intermediate formation between O and O$_{2}$. Volatile Organic Compounds (VOCs) are not shown in Eq \ref{eq:ozoneformation}, but play a significant role in the oxidation of the primary combustion product NO to NO$_{2}$ \citep{Song2011}. In addition to nitrogen oxides (NOx's), VOCs, chlorine \citep{Thornton2010}, solar radiation (SR), relative humidity (RH) and ambient temperature also impact surface ozone formation \citep{Sadanaga2003}.  Local concentrations of O$_{3}$ are further influenced by weather patterns and terrain that disperse the pollutants, precursors, and byproducts \citep{Beck1998}. At night, O$_{3}$ reacts with NO$_{2}$ to form NO$_{3}$ (nitrate radical) \citep{Finlayson1993}:

\begin{equation}
\label{eq:nitrateformation}
O_{3} + NO_{2}\rightarrow NO_{3}+O_{2} 
\end{equation}

The NO$_{3}$ radicals react with NO$_{2}$ to form dinitrogen pentoxide (N$_{2}$O$_{5}$) which in turn forms nitric acid (HNO$_{3}$) through hydrolysis with water or aqueous particles \citep{Song2011}. The acid is finally neutralized by ammonia (NH$_{3}$) to complete the reaction chain \citep{Brown2012}.

Additional contributions to tropospheric O$_{3}$ concentrations come from the stratosphere-troposphere exchange (STE) of stratospheric ozone \citep{Tarasick2008}. The percentage of O$_{3}$ provided by STE at surface levels range from 13\% \citep{Cooper2006} to over 42\% \citep{Lelieveld2000} depending on area and conditions. With so many chemical and transport phenomena taking place throughout the day and night, modeling O$_{3}$ becomes a very complex task even before local terrain, sources and weather patterns are incorporated. Nonetheless, predicting ambient O$_{3}$ concentrations, and particularly concentrations that may exceed air quality standards is important for air managers and at-risk populations.  In cases where O$_{3}$ levels will exceed standards for long periods of time, air managers may issue air quality warnings and even limit industrial and vehicular activities \citep{Kuhlbusch2014, Welch2005}. Improving forecast accuracy provides planning and decision options that can impact receptor health and local economies.

\subsection{Forecasting ozone with machine learning}

Due to the formation process of O$_{3}$, the actual concentration a local population is exposed to may have been generated from precursors emitted hundreds or even thousands of kilometers away \citep{Glavas2011}. Populations living in coastal regions may be exposed to pollutants generated locally but transformed and mixed with other precursors in circulating land-sea breezes \citep{Freeman2017a}. Tropospheric O$_{3}$ is, therefore, a more complex pollutant to estimate than primary pollutants such as SO$_{2}$ or CO.

Many studies have used supervised machine learning techniques, such as artificial neural networks (ANNs) to predict O$_{3}$ time series concentrations \citep{Comrie1997, Dorling2003, Ettouney2009a, Kurt2008, Biancofiore2017}. ANNs have been shown to provide better predictive results than linear models such as Multiple Linear Regression (MLR) and time series models such as Autoregressive integrated moving averages (ARIMA) \citep{Gardner1998, Prybutok2000}. Support vector machines (SVMs) have also been applied to O$_{3}$ prediction scenarios with results that often outperform ANNs \citep{Luna2014, Papaleonidas2013, Singh2013}. 

The benefits of using ANNs include not requiring \textit{a priori} assumptions of the data used for training and not requiring weighting of initial inputs \citep{Gardner1998}. In practice, dimensionality reduction is often used to remove inputs to the model that are not independent and identically distributed (IID) or offer little influence on the overall training. Principal Component Analysis (PCA) is often used to reduce the overall inputs to the model by removing transformed components but provides little variability to the actual number of features required to train \citep{Singh2013, Wang2015a}.

Most of the studies that use feed forward neural network (FFNN) architecture apply a single hidden-layer feed forward neural network architecture trained with meteorological and concentration data and have limited success in forecasting air quality. The canonical FFNN mode; consists of an input layer, a hidden layer, and an output layer. Each layer is constructed from interlinked nodes that generate a value (usually between -1 and 1 or 0 and 1). The individual node model is shown in Figure \ref{fig:SingleANN}. \\
%
\begin{figure}[H]
\centering
\includegraphics[width=\textwidth,keepaspectratio]{images/single-ann.png} 
\caption{Individual node model.}
\label{fig:SingleANN}
\end{figure}
%
The node sums the weighted inputs of the previous layer, sometimes with a bias, and transforms the combined sum with a non-linear activation function, $\sigma$. The node activation equation is given by

\begin{equation}
\label{eq:perceptron}
y= \sigma(wx+b)
\end{equation}

\noindent
where $w$ is an array of weights for the connections between the previous layer and the current layer, $x$ is a vector of input values from the previous layer, and $b$ is a bias value. Common activation functions include the sigmoid, tanh, and relu functions. A general property for activation functions is that they normalize the output and have a continuous first-order derivative that can be used during the back-propagation training process \citep{Goodfellow2016}. The activation functions mentioned earlier are shown in Table \ref{tb:activations} along with their first order derivative and output range.

\begin{table}[H]
\centering
\caption{Common activation functions}
\label{tb:activations}
\begin{tabular}{@{}lccc@{}}
\toprule
\textbf{Name} & \textbf{Equation} & \textbf{Derivative} & \textbf{Output range} \\ \midrule
sigmoid & $\sigma(x) = \frac{1}{1+e^{-x}}$ & $\sigma'(x)=\sigma(x)(1-\sigma(x))$ & $\in 0,1$ \\
tanh & $\sigma(x) = \frac{e^{x}-e^{-x}}{e^{x}+e^{-x}}$ & $\sigma'(x)= 1-\sigma(x)^{2}$ & $\in -1,1$ \\
relu & $\sigma(x) = \left\{\begin{matrix}0, x<0\\ x, x \geq 0\end{matrix}\right.$ & $\sigma'(x) = \left\{\begin{matrix}0, x<0\\ 1, x \geq 0\end{matrix}\right.$ & $\in >0,\infty$ \\ \bottomrule
\end{tabular}
\end{table}
 
More recent studies, however, looked at the limitations of FFNNs, namely the difficulty in choosing a suitable architecture and the tendency to over-fit the training data, leading to poor generalization, particularly in situations where limited labeled data is available \citep{Lu2005, Papaleonidas2013}.  

The predicted outputs in previous air quality forecast studies \citep{Arhami2013} were based on continuous concentration values measured in parts per billion (ppb) or $\mu g/m^{3}$ from single stations. Achar et al. (2011) investigated the intervals between O$_{3}$ exceedances and maxima of daily concentration levels instead of estimating real-time values. Their study of inter-occurrence between peaks was used to determine overall improvements in air quality trends over time as compared to predicting future air quality conditions \citep{Achcar2011}. 

For our validation case study area in Kuwait, several studies were completed that focused on ambient air quality and modeling using ANNs. Abdul-Wahab (2001) used 5 minute measurements of precursors such as CH$_{4}$, Non-Methane Hydrocarbons (NMHCs), CO, CO$_{2}$, Dust, NO, NO$_{2}$, and NO\textsubscript{x}) and meteorological inputs (WS, WD, TEMP, RH, and Solar Radiation) inputs from a mobile site in the Khaldiya residential area to estimate ozone and smog produced (SP)  using a single hidden layer FFNN \citep{AbdulWahab2001}. Al-Alawi and Abdul-Wahab later enhanced their model by applying Principal Component Analysis (PCA) to reduce the dimensionality of the input data \citep{AlAlawi2008}.  Ettouney et al. (2009) used the same inputs as Abdul-Wahab (replacing dust with Methanated Hydrocarbons) and two FFNNs to predict monthly O$_{3}$ concentrations from the Jahra and Um Al Hayman stations. They suggested that O$_{3}$ in Kuwait often comes from outside the local area via long-range transport \citep{Ettouney2009a}. 

\subsection{Deep Learning and time series}
Studies in atmospheric sciences and O$_{3}$ concentration predictions using Deep Learning (DL) have not been as common as single hidden layer FFNN. DL refers to the families of ANNs that have more than one hidden layer or use advanced architectures such as recurrent neural networks (RNNs) and convolutional neural networks (CNNs) \citep{Goodfellow2016}. 

Partially recurrent network models such as the Elman Network (EN) were used with air station inputs to predict ground level concentrations of O$_{3}$ \citep{Biancofiore2015} and PM$_{2.5}$ \citep{Biancofiore2017}. The feedback provides memory to the system when a single input set is fed into the system. Different ANN architectures are shown in Figure \ref{fig:ANNmodels}. The simple node in Figure \ref{fig:SingleANN} is redrawn as a circle for comparison.  The arrows between layers represent synaptic weights that interconnect each node. Figure \ref{fig:ANNmodels} shows that each layer can have different numbers of nodes, however, the number of nodes in deep neural networks (DNNs) hidden layers are usually kept the same for each layer.
%
\begin{figure}[H]
\centering
\includegraphics[width=\textwidth,keepaspectratio]{images/ann-models.png} 
\caption[Different ANN model architectures.]{Different ANN model architectures. (a) simple feed forward neural network, (b) a recurrent (Elman) neural network, and (c) a deep feed forward neural network with multiple hidden layers.}
\label{fig:ANNmodels}
\end{figure}
%

Implementations of procedures such as Long Short Term Memory (LSTM) for RNNs allow network training to take place without having long-term parameters ``explode" or ``vanish" as a result of multiple learning updates \citep{Pascanu2013}. LSTM was first introduced by Hochreiter and Schmidhuber in 1997  to overcome these training issues \citep{Hochreiter1997}. Gomez (2003) was one of the first researchers to use a single layer RNN to forecast maximum ozone concentrations in Austria \citep{Gomez2003}. His model utilized LSTM to outperform other architectures such as ENs. Fan et al (2017) also used an LSTM with an RNN to forecast PM$_{2.5}$ concentrations that outperformed deep feed-forward neural networks and decision trees \citep{Fan2017}.  

DL has recently become popular for many applications due to improvements in training procedures and software libraries such as Theano \citep{Theano2016}, Tensorflow \citep{Tensorflow2015}, and Keras \citep{keras2015}. These libraries have made implementing DL models more accessible, and therefore more accessible to researchers outside of the Machine Learning fields. A discussion of the theory of RNNs is presented in the next section.

\subsection{Time series data}
Air quality data are continuous, multivariate time series where each reading constitutes a set measurement of time and the current reading is in some way related to the previous reading, and therefore dependent \citep{Gheyas2011}. Measured pollutants may be related through photochemical or pre-cursor dependencies, while physical properties limit meteorological conditions. 

Time series are often impacted by collinearity and non-stationarity that also violate independence assumptions and can make forecast modeling difficult \citep{Gheyas2011}. Autocorrelation of individual pollutants show different degrees of dependence to past values.  Correlation coefficients were calculated using the equation
%
\begin{equation}
\label{eq:corr}
Y(\tau)= corr(X(t),X(t - \tau))
\end{equation}
%
\noindent
where $X$ is the input vector of a time step and $\tau$ is the lag (in hours). The correlogram was plotted based on lags up to 72 hours, as shown in Figure \ref{fig:serialcorr}.
%
\begin{figure}[H]
\centering
\includegraphics[width=\textwidth,keepaspectratio]{images/time-o3.png}  %assumes jpg extension
\caption{Correlogram of O$_{3}$ and NOx for 72 hours.}
\label{fig:serialcorr}
\end{figure}
%
The parameters of Figure \ref{fig:serialcorr} show clear diurnal cycles, with O$_{3}$ having very strong relational dependence every 24 hours, regardless of the time delay. In contrast, the dependency of NOx falls rapidly over time, despite peaking every 24 hours. 

Non-stationarity, collinearity, correlations, and other linear dependencies within data are easily handled by ANNs if enough training data and hidden nodes are provided \citep{Goodfellow2016}. More important to time series are the near term history associated with the previous time step. RNNs incorporate near-term time steps by unfolding the inputs over the time sequence and sharing network weights throughout the time sequence. Additionally, the sequence fed to the RNN has fixed order, ensuring that for that individual observation, the sequence follows the order it appeared in, rather than randomly sampled as is the case for FFNN training \citep{Elangasinghe2014}. Previous models using ANNs could assume that some historical essence of the data was incorporated into the weights during updating as long as the training data was fed in temporal order and not shuffled as most categorical applications are \citep{Bengio2012}. Another way of handling sequential data is to use a time-delay neural network (TDNN). This type of architecture takes multiple time steps of data and feeds into the network at the input, using extensions of the input to represent previous states and become the system memory. TDNNs, in modern terminology, are called 1-dimensional CNNs \citep{Goodfellow2016}. They were not considered in this study.

\subsection{Recurrent Neural Networks}
As previously mentioned, RNNs are well suited for multivariate time series data, with the ability to capture temporal dependencies over variable periods \citep{Che2016}. RNNs have been used in many time series applications including speech recognition \citep{Graves2013}, electricity load forecasting \citep{Walid2017} and air pollution \citep{Gomez2003, Fan2017}. RNNs use the same basic building blocks as FFNNs with the addition of the output fed back into the input. This time delay feedback provides a memory feature when sequential data is fed to the RNN. The RNN share the layer's weights as the input cycles through. In Figure \ref{fig:rnn}, $X$ is the input values, $Y$ is the network output, and $H$ is the hidden layers. An FFNN is provided to compare the data flow over time. By maintaining sequential integrity, the RNN can identify long-term dependencies associated with the data, even if removed by several time steps. An RNN with one-time step, or delay, is called an Elman Network (EN) and has been used successfully to predict air quality in previous studies \citep{Biancofiore2015, Biancofiore2017}. The structure of the EN is shown in Figure \ref{fig:ANNmodels}b.

%
\begin{figure}[H]
\centering
\includegraphics[width=\textwidth,keepaspectratio]{images/rnn.png}  %assumes jpg extension
\caption{Architecture of an RNN showing layers unfolding in time n times.}
\label{fig:rnn}
\end{figure}
%

RNNs are trained using a modified version of the back-propagation algorithm called back-propagation through time (BPTT). While allowing the RNN to be trained over many different combinations of time, BPTT is vulnerable to vanishing gradients due to a large number of derivative passes, making an update very small and nearly impossible to learn correlations between remote events \citep{Pascanu2013, Graves2013a}. Different approaches were tried to resolve these training challenges including the use of gated blocks to control network weight updates such as LSTM. LSTMs will be discussed in another section. While RNNs and LSTMs have been around for many years \citep{Hochreiter1997}, their use was limited until recently, in what Goodfellow et al. calls a ``third wave of neural network research". This period began in 2006 and continues to this day \citep{Goodfellow2016}. Like FFNNs, RNNs are trained on loss functions using optimizers to minimize the error. 

\subsection{Loss functions}

The loss function, or cost function, is the function that measures the error between the predicted output and the desired output \citep{Goodfellow2016}. In optimization theory, there are many loss functions that can be used including the Mean Square Error (MSE) and cross entropy (CE) functions being the most popular for machine learning applications.  Selection of the loss function is based on the application. However, the CE is often used for classification applications \citep{Kline2005, Wu2017}. Janocha and Czarnecki (2017) suggested that non-log loss functions were more appropriate for DL based on experimental results \citep{Janocha2017}. In our study, the MSE loss function was used instead of the CE function. The MSE equation is given as 
%
\begin{equation}
\label{eq:MSE}
MSE = \frac{1}{n} \sum_{i=1}^{n} \left( y_{pred} - y_{obs} \right)^{2}
\end{equation}
%
\subsection{Optimizers}
Optimizers provide the method to minimize the loss function and include terms and parameters that determine the amount of incremental changes to the network weights during training. Optimizer terms often include support features such as momentum and regularization. Momentum is used to speed up convergence and avoid local minima \citep{Sutskever2013}, while regularization describes terms that reduce generalization errors \citep{Goodfellow2016}. The most common optimizer used for the back propagation training algorithm used on most neural networks is stochastic gradient descent (SGD). The basic first order SGD equations with a classic momentum (CM) term is given as
%
\begin{equation}
\label{eq:SGDterm}
g_{t+1} = \mu g_{t} - \alpha \nabla f (\theta_{t})
\end{equation}
%
\noindent
where $g$ is the gradient update term, $\mu$ is the momentum factor $(\mu \in (0,1))$, $\alpha$ is the learning rate, and $\nabla f (\theta_{t})$ is the gradient of the loss function for a specific parameter, $\theta_{t}$. The parameter is updated by
%
\begin{equation}
\label{eq:SGDupdate}
\theta_{t+1} = \theta_{t} + g_{t+1}
\end{equation}
%
A major limitation of SGD for training very deep learning networks is the vanishing gradient problem, where the gradient update term becomes so small that no update takes place and the network parameters do not converge. Hinton et al. (2006) introduced greedy layer-wise pre-training in which a network was trained layer by layer and then integrated with SGD when compiled together \citep{Hinton2006}. Since then, other first-order optimizers have been introduced that modify the SGD's basic algorithm by updating the learning rate and momentum terms during the training process \citep{Sutskever2013}.  One such method was to apply a Nesterov accelerated gradient (NAG) \citep{Nesterov1983} term to the SGD gradient update. The NAG update closely resembles the SGD update in Eq \ref{eq:SGDterm} except for the addition of another momentum term in the parameter gradient.
%
\begin{equation}
\label{eq:SGD-NAG}
g_{t+1} = \mu g_{t} - \alpha \nabla f (\theta_{t} + \mu g_{t})
\end{equation}
%
Other algorithms include the adaptive subgradient descent (AdaGrad) optimzer \citep{Duchi2011}, the root mean square propagation (RMSProp) optimizer \citep{Tieleman2012}, the adaptive momentum (Adam) optimizer \citep{Kingma2014}, and the Nesterov adaptive momentum (Nadam) optimizer \citep{Dozat2016}. A summary of how these optimizers differ is shown in Table \ref{tb-optimizers}.
%
\begin{table}[H]
\centering
\caption{Enhanced first order optimizers used for ML models.}
\label{tb-optimizers}
%\scriptsize
%\begin{tabular}{@{}lcc@{}}
\begin{tabular}{@{}p{0.2\linewidth}p{0.6\linewidth}p{0.2\linewidth}@{}}
\toprule
\textbf{Optimizer} & \textbf{Description Summary} & \textbf{Source} \\ \midrule
AdaGrad & Divides the learning rate, $\alpha$, by the $L_{2}$ norm & Duchi (2011) \\
RMSProp & Divides gradient by a running average of its recent magnitude & Tieleman (2012) \\
Adam & Combines CM with RMSProp & Kingma (2014) \\
Nadam & Combines NAG with RMSProp & Dozat (2016) \\ \bottomrule
\end{tabular}
\end{table}    

%
After experimenting with all four of the optimizers in Table \ref{tb-optimizers} and SGD, the Nadam optimizer proved to work the best with our study as described in the next section.

\subsection{Long Short Term Memory}

In order to preserve the memory of the data in the current state of the model, the RNN feeds parameters of its current state to the next state. This transfer can continue for multiple time steps and presented significant training challenges as mentioned earlier. The issue of vanishing gradients that took place during the BPTT updates was largely solved with the implementation of gating systems such as LSTM that allow nodes to forget or pass memory if it is not used, thus preserving enough error to allow updates \citep{Hochreiter1997}. The LSTM uses a series of gates and feedback loops that are themselves trained on the input data as shown in Figure \ref{fig:lstm}. Each node in the LSTM acts like a standard FFNN node, similar to the one in Figure \ref{eq:perceptron}, summing the inputs and applying an activation function at the output. The choice of activation function is another parameter to consider in the LSTM design. Common functions include the $sigmoid$, $tanh$, and $relu$ functions as shown in Table \ref{tb:activations}. In addition to the observations, $X$, input from the recurrent output, $Y_{R}$, representing a time-delayed element of the network, is included for a composite input of 
%
\begin{equation}
\label{eq:Xr}
X_{R}(t) = X(t) + Y_{R}(t-1)
\end{equation}
%
The processed recurrent input, $X_{R}$ feeds into several gates that allow the data to pass, represented by $\Phi$ in the circles. The weights that pass $X_{R}$ to the gate summations are trained as well.
%
\begin{figure}[H]
\centering
\includegraphics[width=.5\textwidth,keepaspectratio]{images/lstm.png} 
\caption[LSTM architecture]{LSTM architecture showing unit time delays (-1), gates and recurrent activation functions ($\sigma$).}
\label{fig:lstm}
\end{figure}

\section{Methodology}

Fixed air monitoring stations are operated throughout Kuwait near residential and industrial areas, but predominantly in the coastal zone areas \citep{Freeman2017a}. For this study, a station using OPSIS differential optical absorption spectroscopy (DOAS) analyzers (www.opsis.se) located near a local college as shown in Figure \ref{fig:Kuwait}. 
%
\begin{figure}[H]
\centering
%\includegraphics[width=.75\textwidth]{images/kuwait.png}  %assumes jpg extension
\includegraphics[width=\textwidth,height=\textheight,keepaspectratio]{images/kuwait.png}
\caption{Location of Kuwait and AMS used in the study.}
\label{fig:Kuwait}
\end{figure}
%
The location is centered between two major highways (5th and 6th Ring Roads) in a concentrated mixed commercial/residential area and north of the Kuwait International Airport. While not near the heavy refineries and industries in southern Kuwait, the site is impacted by land-sea breezes that recirculate emitted pollutants from throughout the Persian Gulf \citep{Freeman2017}. A data set from 1 Dec 2012 to 30 Sep 2014 was used for this study. Parameters available are shown in Table \ref{tb:parameters}.
%

\begin{table}[H]
\centering
\small
\caption{Chemical and meteorological parameters captured at the AMS}
\label{tb:parameters}
\begin{tabular}{@{}cc@{}}
\toprule
\textbf{\begin{tabular}[c]{@{}c@{}}Chemical Analytes\end{tabular}} & \textbf{Meteorological}  \\ \midrule
Nitrous oxide (NO) & Wind direction \\
Ammonia (NH$_3$)& Wind speed   \\
Ozone (O$_3$) & Temperature  \\
Sulfur dioxide (SO$_2$) & \begin{tabular}[c]{@{}c@{}}Atmospheric pressure\end{tabular} \\
Formaldehyde (CH$_2$O) & \begin{tabular}[c]{@{}c@{}}Solar radiation\end{tabular}\\
Nitrogen dioxide (NO$_2$) & Relative humidity   \\
Benzene (C$_6$H$_6$)  &  \\
Toluene (C$_7$H$_8$) &    \\
p-Xylene (C$_8$H$_10$) &    \\
m\_Xylene (C$_8$H$_10$)   &     \\
1,2,3-trimethylbenzene (C$_{6}$H${3}$(CH$_3$)$_3$)   &   \\
o-Xylene (C$_8$H$_10$)   &    \\
\begin{tabular}[c]{@{}c@{}}Ethylene glycol tertiary butyl ether\\   (ETB) (C$_{6}$H$_{14}$O$_{2}$)\end{tabular} &                                                                  \\
Styrene (C$_8$H$_8$)  &   \\
Chlorine (Cl$_2$) &    \\
Carbon dioxide (CO$_2$)  &     \\
Methane (CH$_4$)    &      \\
Hydrogen sulfide (H$_2$S)&    \\
Carbon monoxide (CO) &  \\ \bottomrule
\end{tabular}
\end{table}

%
The local O$_{3}$ air quality standard is 75 ppb measured against an 8 hour moving average \citep{KEPA2017}. The DOAS station recorded O$_{3}$ concentrations hourly, along with the other measured parameters. Measured units in $\mu g/m^{3}$ were converted to $ppb$ using the conversion formula
%
\begin{equation}
\label{eq:gasequation}
C(ppb) = \frac{C(\mu g/m^{3})(R) (T)}{(P) (MW)}
\end{equation}
%

\noindent
where $C(ppb)$ is the gas concentration in $ppb$, $C(\mu g/m^{3})$ is the concentration in $\mu g/m^{3}$, $R$ is the ideal gas constant given as 8.3144 $m^{3}kPa K^{-1}mol^{-1}$, $MW$  is the molecular weight of the gas in $g/mole$ (48.01 $g/mole$ for O$_{3}$), $T$ is the ambient temperature in degrees Kelvin, and $P$ is the atmospheric pressure at ground level in $kPa$. The station receives a prevailing wind from the northwest throughout the year as shown in the windrose in Figure \ref{fig:windrose} generated using WRPLT from Lakes Environmental (\url{https://weblakes.com/products/wrplot/index.html}).
%
\begin{figure}[H]
\centering
%\includegraphics[width=.75\textwidth]{images/paaet-windrose.png}  %assumes jpg
 extension
\includegraphics[width=\textwidth,height=\textheight,keepaspectratio]{images/rnn-windrose.png} 
\caption{Station wind-rose from 2012 to 2014.}
\label{fig:windrose}
\end{figure}
%
Seasonal effects are shown in bivariate polar plot in Figure \ref{fig:bipolarplots}. High O$_{3}$ concentrations occur in the summer months from June to August, but come from the northeast, indicating the transport of pollutants from the coast. Not surprisingly, most of the compliance exceedances take place during this period.
%
\begin{figure}[H]
\centering
%\includegraphics[width=.75\textwidth]{images/paaet-o3seasons.png}  %assumes jpg
%\includegraphics[width=\textwidth,height=\textheight,keepaspectratio]{images/paaet-o3seasons.png} 
\includegraphics[width=\textwidth,height=\textheight,keepaspectratio]{images/rnn-o3seasons.png} 
\caption{Seasonal bivariate polar plots of 1 hour O$_3$.}
\label{fig:bipolarplots}
\end{figure}
%
Average hourly O$_{3}$ and NOx concentrations are shown in Figure \ref{fig:hourlyAveO3}. The two variables are highly inverse correlated ($R^{2}$ = -0.963) with common maxima/minima at 0300, 0600, 1400, and 2200 hrs. The raw hourly data is less correlated ($R^{2}$ = -.576), but still higher than other variables. The NOx peaks correspond to rush hour traffic periods with winds blowing from the 6th and 7th Ring highways in the southwest. O$_{3}$ levels peak in the afternoon, corresponding with solar radiation levels, but there is also a local maximum, or ``morning bump" due to photolyzed chlorine ions reacting with N$_{2}$O$_{5}$ to form NO$_{3}$ as part of the O$_{3}$ formation cycle in Eq \ref{eq:nitrateformation} \citep{Calvert2015}. The formation of NO$_{3}$ radicals was observed to be inversely related to O$_{3}$ concentrations \citep{Song2011}.
%
\begin{figure}[H]
\centering
%\includegraphics[width=.75\textwidth]{images/dailyAveO3.png}  %assumes jpg extension
\includegraphics[width=\textwidth,height=\textheight,keepaspectratio]{images/dailyAveO3.png}
\caption{Hourly averages of 1 hour O$_{3}$ and NOx.}
\label{fig:hourlyAveO3}
\end{figure}
%

\subsection{Building the RNN}
The RNN used in this study was prepared using the Keras machine learning application programming interface (ver 2.0.9) \citep{keras2015} with Theano back-end. Theano is a C\+\+ library that allows mathematical expressions to be calculated symbolically and evaluated using datasets in matrices and arrays \citep{Theano2016}. The architecture used a single RNN layer with LSTM and a single output feed forward node. The output activation function for both layers was the $sigmoid$ function while the activation function for the recurrent nodes was the $tanh$ function. 

The learning rate, $\alpha$, was left at the default value of 0.002 \citep{keras2015}. Other Keras defaults included weight initialization (using a uniform distribution randomizer). Regulation was not used, although a dropout layer was included between the LSTM and the output layer. 

\subsection{Input data preparation}
Each available feature was compared to the maximum possible data range of 16,056 hourly measurements over the observation period. Gaps in data were assumed to be Missing Completely at Random (MCAR) and attributed to maintenance downtime, power failures, and storm-related contamination \citep{Le2007}. Additionally, some data were clearly out of range or had negative readings \citep{Junger2015}.

Data recorded as a 0 was assumed to be censored and converted to the smallest recorded value within the data set of the individual parameter \citep{Rana2015}. Negative and missing data were converted to 0 and identified using a filter mask. Two different single imputation (SI) techniques were used based on the number of consecutive gaps in data. For gaps (g)$<$ 8, the first and last measurement within the gap were used as a Bayesian estimator based on the previous observation to create a linear estimate of the missing data given by 
%
\begin{equation}
\label{eq:impute1}
X_{n} = X_{n-1} + n\Delta
\end{equation}
%
\noindent
where $n$ is the a missing data point in sequence ($0 < n \leq g$), and 
%
\begin{equation}
\label{eq:impute2}
\Delta = \frac{X_{g+1} - X_{0}}{g+1}
\end{equation}
%
For consecutive gaps $>$ 8, the corresponding hourly measurement from the previous and preceding day was averaged.
%
\begin{equation}
\label{eq:impute3}
X(t) = \frac{X_{t+24} - X_{t-24}}{2}
\end{equation}
%
The value of 8 consecutive gaps was determined by comparing the root mean square error (RMSE) of the original data with data generated from the different statistical (imputed data) methods on artificially generated gaps \citep{Junninen2004}. The first method's error varied with gap size, while the second method had a higher, static error. The RMSE for O$_{3}$ and NOx using the first method is shown in Figure \ref{fig:impute-rmse} along with the intersection of the 2nd method.
%
\begin{figure}[H]
\centering
%\includegraphics[width=.75\textwidth]{images/impute-rmse.png}  %assumes jpg extension
\includegraphics[width=\textwidth,height=\textheight,keepaspectratio]{images/impute-rmse.png}
\caption{RMSE of O$_{3}$ and NOx from consecutive gaps of data using the first imputation method.}
\label{fig:impute-rmse}
\end{figure}
%
Features with more than 50\% missing data, like RH and Chlorine (Cl$_{2}$), were discarded. The remaining variables from Table \ref{tb:parameters} had few missing data points, ranging from 41 missing points out of 16,053 (0.3\%) for WS, WD, and Temperature, to 137 missing points (0.9\%) for NH$_{3}$. The available data used for this study is larger than datasets used in other studies that had up to 16\% missing data \citep{Taspinar2015}. Missing data adds noise to the training set and is often used to improve generalization and prevent overfitting the network to meet the training dataset. Adding dropouts, or intentional data removal is often used during network training for this reason \citep{Srivastava2014}.

\subsubsection{Cyclic and Continuous Data}
WD and time of day were converted into representations that preserved their cyclic nature. Wind direction was converted into sine and cosine components \citep{Arhami2013}. Other parameters were transformed and scaled between values of 0 and 1 \citep{Chatterjee2017} using the \textbf{MinMaxScaler} function in the Python Sci-Kit pre-processor library \citep{scikit2011}. Overall, 25 features were prepared for initial training. These included the parameters measured in Table \ref{tb:parameters} with the addition of sine and cosine components for wind direction. 
 
\subsubsection{Feature selection using Decision Trees}
Before training the RNN, features were reviewed to reduce input dimensionality by training multiple decision trees on the data sets and prioritizing features using the feature importance metric calculated during classifier training with the \textbf{DecisionTreeClassifier}, also in the Sci-kit library \citep{scikit2011}. Decision trees and random forest classifiers have been used to reduce input dimensions for sensors \citep{Cho2011} and data classification competitions, outperforming other methods such as PCA and correlation filters \citep{Silipo2014, Al-Alawi2008}. Decision trees are a supervised learning algorithm that recursively partitions inputs into non-overlapping regions based on simple prediction models \citep{Singh2013, Loh2011}.  Decision trees do not require intensive resources to train and evaluate, and keep their features, unlike PCA that transforms input variables into linear combinations based on the singular value decomposition (SVD)of the total data set's covariance matrix \citep{Wang2016}. 

While PCA is a form of unsupervised learning that allows dimensionality reduction by removing the number of transformed components fed to the input, decision trees identify which raw variables offer less impact so that they can be removed from the data collection stream. Reducing features can improve future efforts required to collect, clean and prepare datasets. 

Using PCA for model input also limits the inclusion of new data that may become available on real-time systems. If a model is trained on transformed principal components only, any new data must also be transformed, thus changing the historical data set. Using a linear transformation method that only changes the individual observation and not the entire data set is more practical for time series applications where new data is expected to increase. 

Binary classification trees, a type of decision tree used for categorical separation,  were trained to predict exceedances of 8 hr ave O$_{3}$ over 1 hr to 12 hr horizons. Individual observations were first scaled using the \textbf{MinMaxScaler} function based on the equation
%
\begin{equation}
\label{eq:MaxMin}
x_{scaled} = \frac{x_{i} - x_{min}}{x_{max} - x_{min}}
\end{equation}
%
\noindent
where $x_{max}$ and $x_{min}$ are the maximum and minimum values in the dataset respectively. It can be argued that this method also suffers from legacy biasing like PCA in that the system retains $x_{max}$ and $x_{min}$ in order to restore transformed data to original scale, similar to a key for encryption decoding. If new data is included that exceeds the $x_{max}$ / $x_{min}$ values, the data needs to be reprocessed using the new points. Since we are already working with an historical data set and not using real-time data, there is no need to incorporate this issue. However, even if we were using real-time data, we could safely assume that any value measured that exceeded $x_{max}$ in our historical set was also an extreme point and could be accounted for in the model as a value $>1$. 

Eighty percent (80\%) of the scaled data was divided into a training set with 20\% reserved for testing.  In larger data sets, a percentage of the total data is often reserved to allow parameter optimization before training in order to reduce time and system resources. Additionally, for FFNNs, training often takes hundreds or thousands of epochs, where an epoch is a complete training cycle that uses all training sets. With our relatively small data set and fast training using the LSTM, reviewing system performance using the full training set did not take much time and therefore did not require reserving a subset.

The decision tree classified output exceedances as either 0 (less than the exceedance standard of 75 $ppb$) or 1 (exceeds the standard). The overall accuracy of each horizon was measured using the \textbf{accuracy\_ score} function in Sci-kit which calculated the standard error of exact matches of the observed output with the predicted \citep{Raschka2016}.  Other classifiers were evaluated as well, including the Support Vector Machine (SVM) and Random Forest classifiers. The decision tree in classification mode using ``gini" criteria to measure the data split at each decision node proved to be the most accurate. The importance of each feature was computed using the Sci-kit \textbf{feature\_ importances\_ function} that normalizes the total reduction of the criterion brought by individual features. Results of individual feature importance from classifying O$_{3}$ exceedances at different horizons are shown in Figure \ref{fig:importance}. The computed values are unitless and displayed in relative importance to each other.
%
\begin{figure}[H]
\centering
%\includegraphics[width=.75\textwidth]{images/sumfactors.png}  %assumes jpg extension
\includegraphics[width=\textwidth,height=\textheight,keepaspectratio]{images/sumfactors.png}
\caption[Decision tree importance factors]{Sum of importance factors from decision trees predicting 8 hr O$_{3}$ from 1 to 12 hours.}
\label{fig:importance}
\end{figure}
%
It is interesting to note in Figure \ref{fig:importance} that the two NOx components (NO and NO$_{2}$) have relatively little importance to the the output of the decision tree.  The results do not represent correlation between the  features, but rather impact to the variance of output and a measure of independence. As noted in earlier sections, ozone is complex secondary product of which NOx is only part of the reaction component. The measured VOCs in the figure have more influence on the output prediction. 

\subsection{Output data preparation}
The RNN output was trained to predict the 8 hr moving average of measured 1 hr O$_{3}$. To predict future values, the calculated values were shifted in time based on the desired horizon so that input observations $X(t=0)$ was trained on $Y(t=12)$ if the prediction horizon was 12 hours. Output data was generated from 8 hr moving averaged O$_{3}$ calculated from measured 1 hr O$_{3}$ concentrations at each station. The first seven hours of both the input and output training data set was then discarded. 

\subsection{Tensor preparation for RNN input data}
Data sets provided to the RNN were converted into 3 dimensional tensors based on the sample size of data. The sample size was based on the number of look-back elements within the RNN, as compared to an observation which represented one row of the original data set, $X$.  The transformation of the original 2 dimensional data set $X$ is illustrated in Figure \ref{fig:tensor-tables} using Python notations. Assuming $X$ is a data set of input data (for training or testing the RNN) with $n$ observations and $p$ variables, the total number of elements is the product of $n$ and $p$, or 20 elements for the 5 x 4 data set in the figure. A tensor ($T$) is created with dimension ($s, l, p$) where $l$ is the number of look-back time steps, $s$ is the \# of samples in the batch, given as $n - l$. The total number of elements within $T$ is $s*l*p$. In the example of Figure \ref{fig:tensor-tables}, the dimensions of $T$ are $s = 5 - 2 = 3$, $l = 2$, and $p = 4$.    
%
\begin{figure}[H]
\centering
%\includegraphics[width=.75\textwidth]{images/tensor-tables.png}  %assumes jpg extension
\includegraphics[width=\textwidth,height=\textheight,keepaspectratio]{images/tensor-tables.png}
\caption{Process of converting data input columns into a Tensor for training the RNN.}
\label{fig:tensor-tables}
\end{figure}

\section{Results}

\subsection{Final parameter selection}
The model was trained on 80\% of the processed data and tested against 20\% of the total available data. Because of the Tensor formation for input, the actual number of samples provided for training and testing was based on the look-ahead horizon and number of recurrent (look-back) units of the individual run. The farther out the prediction, the fewer samples were available because of the time shifting required. The total amount of samples available for training and testing could be calculated as total samples = $(16,035 - h)$ where $h$ is the prediction horizon (as an integer value $>$ 1). 

The number of training epochs was limited after reviewing training error values up to 20 epochs for look-ahead horizons of 24 hrs, 36 hrs, and 48 hrs as seen in Figure \ref{fig:horizon-loss}. An optimum number of 10 epochs was used for later model runs as it minimizes the training error without overfitting which begins to take place after 12 epochs, especially in Figure \ref{fig:horizon-loss}a. 
%
\begin{figure}[H]
\centering
%\includegraphics[width=.5\textwidth]{images/horizon-loss.png}  %assumes jpg extension
\includegraphics[width=\textwidth,height=\textheight,keepaspectratio]{images/horizon-loss.png}
\caption[Loss function errors of training and test data sets]{Loss function errors for training and test data sets for different horizons at (a) 24 hr, (b) 36 hr, and (c) 48 hr.}
\label{fig:horizon-loss}
\end{figure}
%

\subsection{Performance measures}

Final parameter selection and performance were measured by Mean Absolute Error (MAE) given by 
%
\begin{equation}
\label{eq:MAE}
MAE = \frac{1}{n}\sum^{n}_{i=1} \left | y_{obs_{i}}- y_{pred_{i}} \right |
\end{equation}
%
and Root Mean Square Error given by
%
\begin{equation}
\label{eq:RMSE}
RMSE = \sqrt{\frac{1}{n}\sum^{n}_{i=1} \left ( y_{obs_{i}}- y_{pred_{i}} \right )^{2}}
\end{equation}
%
MAE and RMSE are widely used measures of continuous variables with RMSE criticized for over-biasing towards large errors \citep{Chai2014, Willmott2005}. Both metrics were calculated for comparison; however MAE is used more often for descriptive analytics.

\subsection{Impact of features and parameters on results}

The network trained very well with all 25 input features from Table \ref{tb:parameters}. An example of the predicted results compared with the observed measurements (8 hr ave O$_{3}$) over a 24 hr horizon is shown in Figure \ref{fig:example24}.
%
\begin{figure}[H]
\centering
\includegraphics[width=\textwidth,height=\textheight,keepaspectratio]{images/example24.png}
\caption{Results of training an RNN with a 24 hr horizon.}
\label{fig:example24}
\end{figure}
%
The MAE for this scenario is 0.41 ppb during training and 0.37 ppb during testing. The residuals of this scenario are shown in Figure \ref{fig:normalresiduals} where they show a Normal distribution tendency (skewness = 0.411, kurtosis = 3.94, where Normal is 0 and 3, respectively) with a positive bias given by the distribution mean of 1.632 ppb. This is consistent with \ref{fig:example24} that shows the model slightly under-predicting.

%
\begin{figure}[H]
\centering
\includegraphics[width=\textwidth,height=\textheight,keepaspectratio]{images/normalresiduals.png}
\caption{Distribution of residual test errors for a 24 hr horizon network.}
\label{fig:normalresiduals}
\end{figure}
%

The results of the decision tree analysis in Figure \ref{fig:importance} showed that many features could be removed without impacting network performance. Features were removed based on the order of least importance in groups of 5 until the most prominent feature remained. The results in Figure \ref{fig:features} show that overall training error improves with fewer inputs. By removing input features, the system complexity is also reduced, allowing the network to train easier. While providing better training results, reducing feature inputs also makes it easier to overfit on the training data.
%
\begin{figure}[H]
\centering
%\includegraphics[width=.75\textwidth]{images/features.png}  %assumes jpg extension
\includegraphics[width=\textwidth,height=\textheight,keepaspectratio]{images/features.png}
\caption{Training error associated with feature reduction on network prediction.}
\label{fig:features}
\end{figure}
%
Based on the training error curves in Figure \ref{fig:features}, the five feature dataset was used for evaluation because it provided stable errors over the prediction horizons of interest. The features used were (in order of importance) 8 hr ave O$_{3}$ in $ppb$, 1 hr O$_{3}$ in $ppb$, SR, the cosine of WD, and CH$_{4}$.

Parameter sensitivity analysis was performed on a model with default values shown in Table \ref{tb:default-parameter}.
%

\begin{table}[H]
\centering
\caption{Default values for parameter sensitivity analysis.}
\label{tb:default-parameter}
\begin{tabular}{@{}lc@{}}
\toprule
\textbf{Parameter} & \textbf{Default Value} \\ \midrule
Input features & 26 \\
Prediction horizon (hours) & 24 \\
Look back nodes & 26 \\
Samples/batch & 72 \\
Dropout factor & 0.2 \\ \bottomrule
\end{tabular}
\end{table}
%
All other parameters were held constant as an individual parameter was varied. The prediction horizon value of 24 hrs was held constant throughout all runs shown in  Figure \ref{fig:parameters}. In all cases, the error measurements, MAE and RMSE showed similar forms, despite the RMSE having a consistently higher value, as expected. The prediction horizon of the model using the 5 feature data set is shown in Figure \ref{fig:predictionhrs}.

%
\begin{figure}[H]
\centering
%\includegraphics[width=.75\textwidth]{images/predictionhrs.png}  %assumes jpg extension
\includegraphics[width=\textwidth,height=\textheight,keepaspectratio]{images/predictionhrs.png}
\caption{Prediction horizons using 5 features and default parameters.}
\label{fig:predictionhrs}
\end{figure}
%
As the prediction extends further into the future ($>$ 80 hrs), the training error climbs rapidly, while the test error appears to level off. The model began to overfit by this point and predictions past that range were considered to be unreliable.

The results in  Figure \ref{fig:parameters}a show training error for different prediction horizons over several parameters. The parameter that influences the model performance the most is the number of look-back nodes in relation to the prediction horizon. A horizon value of + 2 provides the lowest errors while adding additional nodes increases the model complexity and makes training more challenging. 

Samples/batches in Figure \ref{fig:parameters}b show relatively little error variance until many samples are included ($>$75 samples/batch). While more samples per batch are preferred to reduce training time, too many create bias in the loss function as the overall average of each sample reduces chances for updates.

The number of recurrent, or look back nodes, in relation to the prediction horizon, was considered in  Figure \ref{fig:parameters}b. Both training and test results are minimized at 26, the horizon value + 2. This value was consistent with other horizon prediction values such as 36 and 48. As more look back nodes are added, the error increases.

Finally, the use of dropout is recommended to improve generalization of the model and reduce overfitting \citep{Gal2016}. For this model, dropout was applied only between the output of the LSTM layer and the FF output layer. The error shows reasonably good optimization at around 0.2. The errors level out at higher rates at around 0.35. The default values were the optimum parameters based on the results shown in Figure \ref{fig:parameters}.
%
\begin{figure}[H]
\centering
%\includegraphics[width=.75\textwidth]{images/parameters.png}  %assumes jpg extension
\includegraphics[width=\textwidth,height=\textheight,keepaspectratio]{images/parameters.png}
\caption[Impact of model parameters on training error]{Impact of (a) batch samples, (b) Look back nodes, and (c) dropout factor parameters on training errors in the model.}
\label{fig:parameters}
\end{figure}
%

An LSTM model has many more variables that can be optimized compared to other models. The RNN used in this study with 25 input features had 16,485 update-able parameters, of which 16,432 were in the LSTM layer alone. As a comparison, an FFNN with 3 hidden layers (5 layers total), bias on all layers, and the same 25 feature inputs, had only 2,107 parameters. Nonlinearities are further introduced during the training phase in which the derivative of the activation function for each layer is used to influence the weight updates. It is therefore difficult to fully explain the mechanisms driving the output results of complex DL models. In order to ensure the model is working, the output must be compared with known results and parameters adjusted to optimize performance.

\subsection{Comparison to previous studies}
Previous studies mentioned in Section 1 used RMSE and other error measurement methods than MAE. The 3 studies that used RMSE were compared to our results with an LSTM network. Luna et al. (2014) used SVMs and FFNNs \citep{Luna2014}. Feng et al. (2011) used a multi-layered system that included an SVM and a genetic algorithm stabilized FFNN \citep{Feng2011}. Wang and Lu used an FFNN with a particle swarm optimization \citep{Wang2006}. All studies, except Gomez (2003), used PCA to pre-process the data. Comparing the results of the RNN to these previous studies gives an initial impression that the RNN has an order of magnitude improvement over the best FFNN or SVM models as shown in Table \ref{tb:compare}.
%

\begin{table}[H]
\centering
\caption{Comparison of LSTM RNN test data results to previously published results.}
\label{tb:compare}
\resizebox{\columnwidth}{!}{%
\begin{tabular}{@{}lcccc@{}}
\toprule
\textbf{Source} &  \textbf{Prediction horizon} & \textbf{Results (RMSE)} & \textbf{LSTM (RMSE)} \\ \midrule
Luna et al. (2014) & 1 hr & 6.3 - 12.3 & 0.8 \\
Feng et al. (2011) &  12 hr & 5.5 - 86.9 & 1.5 \\
Wang and Lu (2006) & 24 hrs & 7.9 - 11.2 & 2.5 \\ 
Gomez et al. (2003)& 24 hrs & 6.9 - 9.9 & \\ \bottomrule
\end{tabular}
} %end resize
\end{table}
%
The results cannot however be directly compared because they were based on different data sets. While the other studies used complex and hybrid architectures along with complicated pre-processing, the RNN model pre-processing was very simple after the features were prioritized using a decision tree. The RNN and LSTM are themselves complex algorithms with many internal parameters that undergo training and updates.

\subsection{Comparison to different models}

In order to evaluate how the RNN model compared to other forecasting models using the same data set, a comparison was made for a 24 hr prediction using an FFNN with three hidden layers, and an ARIMA model. The FFNN was built with the Keras library using $relu$ activation functions in the hidden layers and a $sigmoid$ function for the output.  The model was allowed to train for 1,000 epochs using all the parameters from Table \ref{tb:parameters}. The number of nodes in the hidden layer was based on the estimated \# of hidden nodes = ($SF$ * \# of input nodes) + \# of output nodes where $SF$ is a scaling factor between 0.5 and 1 \citep{Papaleonidas2013}. For an $SF$ of 0.75, the number of nodes is 20 nodes in each hidden layer. An MSE loss function and Nadam optimizer were also used to build the model. The ARIMA model was built using the \textbf{auto.arima} function in R (ver 3.4.3) \citep{Hyndman2013} and fitted on the 8 hr ave O$_{3}$ only. The formatting parameters, $p$, $d$, and $q$ were calculated as 3, 0, and 0, respectively. The results are shown in Figure \ref{fig:diffmodels}.

%
\begin{figure}[H]
\centering
\includegraphics[width=\textwidth,height=\textheight,keepaspectratio]{images/modelcompare.png}
\caption{Comparison of different model forecasts over a 24 hr period}
\label{fig:diffmodels}
\end{figure}
%

The continuous nature of the RNN and ARIMA predicted curves are in contrast to the FFNN predicted results due to the lack of memory inherent within the FFNN algorithm. While not able to predict as the time series as well as the RNN, the ARIMA model allows the previous result to influence the next time step. The FFNN model requires all necessary information to be provided within the input sample and network weights, allowing for better prediction results than the ARIMA, but a non-continuous form. 

\section{Conclusions}
This research is one of the first of its kind to use Deep Learning techniques to predict of air quality time series events. This new methodology produced very good results using our validation data set. A recurrent neural network with LSTM was trained on time series air pollutant and weather data from an air monitoring station in Kuwait to predict 8 hour average O$_{3}$ over different prediction horizons. Missing data and censored data were replaced using a first-order imputation technique that accounted for the sequential influence of previous readings for small gaps ($<$ 8) and seasonal effects for larger gaps. 

A decision tree was used to prioritize the most influential features for training by categorizing pollution exceedances using the input parameters. Prioritizing and removing less important features allows for real-time observations to be fed into the model without transforming large blocks of data as is required when using principal components or wavelets. New observations need only be scaled by normalizing or standardizing with the scaling values calculated from historical data sets. 

A sensitivity analysis of key parameters showed that the network could be tuned for optimal performance. Measurements of the performance, in terms of observed and predicted results, were consistent in form, but the RMSE was always biased higher than the MAE measurement. Either measurement would have produced the same conclusions based on observation of local minima and maxima regardless of the error value. Error increased with the complexity of the network, even with reduced features. This ``Curse of Dimensionality" led to overfitting of the model, reducing the ability to generalize if new data sets were introduced. Slight overfitting is not a problem for time series data that follow predictable cyclic patterns and the main output product of interest is when that pattern goes higher than a set limit.

While the results cannot be directly compared to other studies because different data sets were used, the results should not be dismissed either. Comparing the same data set results to other common forecasting models such as ARIMA and a multi-layer FFNN shows that the RNN does perform significantly better. The complexity of RNN implementation has been dramatically reduced with the use of the Keras developmental library, allowing non-computer scientists the ability to use DL without the coding overhead. The LSTM model provided very good results for this case and can be applied to other environmental time series challenges such as forecasting wide area pollution exceedances from multiple stations and multiple pollutants. LSTMs could also be effective in predicting individual source emissions or modeling source apportionment under different criteria. 

Reducing features and optimizing parameters assisted with lowering error of both training and test sets. Initial runs using local data showed excellent results compared to performance from FFNNs, even with the inclusion of complex pre-processing of input data and architecture of the model. The relative ease of model structure in the programming code is misleading though. The Keras and Theano libraries are some of the most advanced and complex libraries available in the Python community. 

The underlying errors within the model implementation may not be resolved or even quantified. However, they are still useful tools for rapid prototyping and architecture validation. Using the data sets of the sources listed in Table \ref{tb:compare} with our RNN would be a more direct way to prove which method works better.

%---------------------------------------------------------------
%------------------------End of Chapter----------------------
\bigskip
\begin{center}
END OF CHAPTER
\end{center}