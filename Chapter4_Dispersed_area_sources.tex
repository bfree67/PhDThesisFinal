\chapter{Dispersed area sources}

\section{Introduction}

Air emission inventories are used to identify air pollutants by source and process type over a given period (usually annually) and within assigned areas.  Many countries require an annual inventory of designated air pollutants such as the United States \citep{USEPA2014} and Kuwait \citep{KEPA2017}.  The United Nations requires signatories of the Kyoto Protocol to complete annual inventories of GHGs \citep{Paciornik2006}.  Because of the scale of emission inventories, actual annual emissions for individual sources are not available due to the lack of individual stack monitoring. In order to estimate emissions, parametric equations are developed that include factors based on the amount of fuel or feedstock consumed over the reporting period of a particular process.  An equation is developed for each pollutant from each fuel type and each process.  The quality of the calculated pollutant emission factors varies significantly depending on the degree of research applied to the fuel and process.  The most common source of published emission factors is the USEPA’s AP-42 series that includes over 27,200 different factors for pollutants from various industrial, commercial and residential processes \citep{USEPA1995}.

Different types of inventory methods exist based on the level of input data available. The Intergovernmental Panel on Climate Change (IPCC) recognizes three types of inventories.  Tier 1 inventories use IPCC published emission factors and nationally available fuel data to estimate individual pollutants.  Tier 2 uses country-specific emission factors, while Tier 3 inventories use process-specific models \citep{Paciornik2006}. 

While several studies looked at indoor air quality in rooms with active $nargyla$ pipes and smokers \citep{Fromme2009, Moon2015, Mulla2015}, specific emissions factors for $nargyla$ smoking suitable for emission inventories were not discovered during the literature review. These studies focused on measuring chemical concentrations as compared to rates of emission. Emission rate studies on individual elements of the smoking process, such as charcoal burning and tobacco burning were identified and used to create composite factors as described in the next sections. 

\subsection{Emissions from $nargyla$ Pipes}
The popularity of $nargyla$ pipes throughout the world has grown in popularity over the last decade \citep{Chaouachi2009, Monzer2008}, especially among young people, where over 30\% of university students in some countries reported regular use \citep{Eissenberg2009}. This increase is a concern for public health organizations and has led to many studies looking at toxic chemical exposure from smoke generated by $nargyla$ pipes \citep{Daher2010, Eissenberg2009, Monzer2008, Sepetdjian2010, Shihadeh2005}.  Chemical analysis has shown that $nargyla$ pipe smoke has higher levels of CO, polycyclic aromatic hydrocarbons (PAHs), benzene, and other Non-Methane Organic Compounds (NMOCs) than smoking cigarettes during the same period.

The basic $nargyla$ water pipe, as shown in Figure \ref{figng1:pipe} consists of a bowl containing sweetened tobacco often referred to as $sheesha$ or $ma’saal$.  The $sheesha$ comes in many flavors and does not have the same lingering odors as other tobacco products, such as cigarettes and cigars.  The tobacco is separated by a small air gap from the glowing charcoal by aluminum foil.  The base is connected to a long tube that ends in a water-filled base.  A hose attached to the base, but above the water level, allows the user to draw hot air from the charcoal onto the $sheesha$, enabling it to singe and smoke.  The smoky air is forced through the water that cools the smoke and allows the user to inhale it.  Because the $sheesha$ is not directly burned, it can be smoked for extended periods of time, sometimes over 1 hr per serving.  Studies have shown that the $sheesha$ itself contributes relatively little to the smoke other than the flavor and particulate matter with the primary source of emissions (over 95\%) coming from the combustion of the charcoal \citep{Sepetdjian2010}.

%
\begin{figure}[H]
\includegraphics[width=\linewidth,keepaspectratio]{images/ng1.png} 
\caption{$Nargyla$ pipe with labeled components.}
\label{figng1:pipe}
\end{figure}
%
While most studies on $nargyla$ smoking have focused on the health effects of exposure to the toxic gases, no studies have looked at the overall contribution of smoking to greenhouse gases (GHGs) and ambient air pollution levels.  The primary contributor of air pollutants is the combustion of charcoal used to heat the $sheesha$ tobacco, which produces CO$_{2}$), methane (CH$_{4}$), and nitrous dioxide (N$_{2}$O) in addition to hazardous air pollutants mentioned previously.  

Sheesha tobacco is a mix of cut tobacco and syrups such as molasses, honey or glycerol with fruit and spice flavors \citep{Chaouachi2009}.  Unlike other tobacco products, $sheesha$ is not burned directly. Hot air from the burning charcoal provides a heat source that singes the tobacco but does not directly burn it, creating a smoke composed of particulates and gases \citep{Daher2010} that is filtered by bubbling through water.  The water removes approximately 50\% of the particulates but does not remove the gaseous components \citep{Becquemin2008}. There is a variety of $sheesha$ smoking called saloom in which the charcoal is placed directly on the tobacco. This tobacco is different from most $sheesha$ in that it is drier and does not have added flavors or syrups. The effects of $saloom$ are not considered in this study as relatively few users opt for this method of smoking.

Emissions measured in some $nargyla$ emission exposure studies used chemically processed, fast lighting coals that are not used in the Persian Gulf countries \citep{Daher2010, Shihadeh2005}.  These types of coals have accelerants mixed with the charcoal such as potassium chlorate (CAS Number 3811-04-9) \citep{MIC2012} to allow easier lighting as compared to the traditional lump charcoal that requires an external heat source to initiate combustion.  The presence of accelerants provides an additional chemical source for emissions but also reduces the overall amount of charcoal used by a caf\'e or restaurant in that a supply of charcoal must otherwise be started and re-supplied when older coals are consumed. Studies looking at different types of $nargyla$ charcoal showed that lump charcoal had more than six times less PAH emissions than charcoals with accelerants \citep{Sepetdjian2010}.  Only lump charcoal is used in Kuwait and therefore is the focus of this study.

Charcoal is manufactured through the pyrolysis of wood until all water, and volatile compounds are removed.  The remaining mass is often 70\% pure carbon, allowing a consistent burn rate with very little smoke. Charcoal is an ideal heat source that is often used for cooking indoors and industrial heating.  Charcoal for $nargyla$ used in the Middle East is made from lemon and orange wood or grape vines in eastern Africa.  Charcoal made from coconut husks is also common for residential use.  Manufacturing charcoal is an emissions-intensive process that generates significant amounts of both HAPs and GHGs \citep{Lacaux1994}.  However, these emissions are not considered in this study.

\section{Methodology}
To determine the annual emissions, both GHGs and HAPs for $nargyla$ pipes, three components were required: 
\begin{itemize}
    \item The amount and type of emissions generated from the charcoal per mass consumed
    \item The amount and type of emissions from the singed $sheesha$ generated per mass consumed
    \item The estimated annual raw consumption of $sheesha$ tobacco and charcoal by commercial caf\'es and restaurants 
\end{itemize}
\noindent
that serve $nargyla$. 

To estimate emission types and rates for $sheesha$ and charcoal combustion, values from the related published research were used \citep{Akagi2011, Bhattacharya2002, Paciornik2006, Sepetdjian2010, USEPA1995}.  Smoking session parameters, such as sitting durations and puff lengths, were also taken from published works \citep{Eissenberg2009, Fromme2009, Mulla2015}.  Estimating the amount of $sheesha$ and charcoal used per year in each caf\'e and restaurant was based on interviews with managers, with the estimated quantities assumed for all venues. Using this approach, we can apply the Central Limit Theorem (CLT) and assume a Normal distribution for analysis.  

Each variable mentioned represents an independent parameter that can vary based on the type of $sheesha$ tobacco, charcoal, and user habits.  This investigation employed the Monte Carlo Analysis (MCA) as an efficient and accepted way to evaluate variance created by multivariate air quality conditions \citep{Freeman2017a, McVoy1979, Tan2014}.  

\subsection{Emissions from $nargyla$ pipes}

Sheesha tobacco is a mix of cut tobacco and syrups such as molasses, honey or glycerol with fruit and spice flavors \citep{Chaouachi2009}.  Unlike other tobacco products, $sheesha$ is not directly burned. Hot air from the burning charcoal provides a heat source that singes the tobacco, creating a smoke composed of particulates and gases \citep{Daher2010} that is filtered by bubbling through water.  The water removes approximately 50\% of the particulates but does not remove the gaseous components \citep{Becquemin2008}. There is a variety of $sheesha$ smoking called saloom in which the charcoal is placed directly on the tobacco. This tobacco is different from most $sheesha$ in that it is drier and does not have added flavors or syrups. The effects of $saloom$ are not considered in this study as relatively few users opt for this method of smoking.

Emissions measured in some $nargyla$ emission exposure studies used chemically processed, fast lighting coals that are not used in the Persian Gulf countries \citep{Daher2010, Shihadeh2005}.  These types of coals have accelerants mixed with the charcoal such as potassium chlorate (CAS Number 3811-04-9) \citep{MIC2012} to allow easier lighting as compared to the traditional lump charcoal that requires an external heat source to initiate combustion.  The presence of accelerants provides an additional chemical source for emissions but also reduces the overall amount of charcoal used by a caf\'e or restaurant in that a supply of charcoal must otherwise be started and re-supplied when older coals are consumed. Studies looking at different types of $nargyla$ charcoal showed that lump charcoal had more than six times less PAH emissions than charcoals with accelerants \citep{Sepetdjian2010}.  Only lump charcoal is used in Kuwait and therefore is the focus of this study.

Charcoal is manufactured through the pyrolysis of wood until all water, and volatile compounds are removed.  The remaining mass is often 70\% pure carbon, allowing a consistent burn rate with very little smoke. Charcoal is an ideal heat source that is often used for cooking indoors and industrial heating.  Charcoal for $nargyla$ used in the Middle East is made from lemon and orange wood or grape vines in eastern Africa.  Charcoal made from coconut husks is also common for residential use.  Manufacturing charcoal is an emissions-intensive process that generates significant amounts of both HAPs and GHGs \citep{Lacaux1994}.  However, these emissions were not considered in this study.

Charcoal, once heated and glowing, burns with a surface temperature of approximately 800$^{o}$ C without additional air blowing on it \citep{Evans1977}.  Ash is formed on the surface as the combustion works its way into the fuel source. Major components of the charcoal combustion are CO$_{2}$, CO, and CH$_{4}$.  Emission factors collected from various sources are summarized in Table \ref{tb1:initialfactors}.  The factors were converted from published units (such as pounds of pollutant/ton of feedstock) to grams of pollutant/kg of feedstock (g/kg).

Only gaseous phase pollutants are quantified. Particulate matter is not considered in this study because of the assumption that the particulates stay in the room or are filtered by the ventilation system. In either case, they are assumed to not contribute to the local ambient air quality.
%
\begin{sidewaystable}  % table rotated 90 degrees
\begin{table}[H]
\centering
\caption{ Emission factors in $g/kg$}
\label{tb1:initialfactors}
\resizebox{\columnwidth}{!}{%
\begin{tabular}{@{}rcccccccccccc@{}}
\toprule
 & \multicolumn{3}{c}{\textbf{IPCC}} & \multicolumn{5}{c}{\textbf{Literature}} & \multicolumn{4}{c}{\textbf{AP-42  (USEPA, 1995)}} \\ 
 & \multicolumn{3}{c}{\textbf{Paciornik (2006)}} & \multicolumn{2}{l}{\textbf{Akagi (2011)}} & \multicolumn{2}{l}{\textbf{Bhattacharya (2002)}} & \multicolumn{1}{l}{\textbf{Sepetdijan (2010)}} & \multicolumn{1}{l}{\textbf{Mixed}} & \multicolumn{1}{l}{\textbf{Leaves}} & \multicolumn{1}{l}{\textbf{Weeds}} & \multicolumn{1}{l}{\textbf{Cigarettes}} \\
\textbf{Compound} & \textbf{Expected} & \textbf{Lower} & \textbf{Upper} & \textbf{Expected} & \textbf{variation} & \textbf{Low} & \textbf{High} & \textbf{Expected} & \textbf{Expected} & \textbf{Expected} & \textbf{Expected} & \textbf{Expected} \\ \midrule
Carbon Dioxide (CO$_{2}$) & 3304 & 1415.5 & 7656 & 2385 &  & 2155 & 2567 &  &  &  &  &  \\
Carbon Monoxide (CO) &  &  &  & 189 & 36 & 35 & 198 &  &  & 56 & 42.5 &  \\
Methane (CH$_{4}$) & 5.9 & 1.043 & 34.8 & 5.29 & 2.42 & 6.7 & 7.8 &  &  & 6 & 1.5 &  \\
Acetylene (C$_{2}$H$_{2}$) &  &  &  & 0.42 &  &  &  &  &  &  &  &  \\
Ethylene (C$_{2}$H$_{4}$) &  &  &  & 0.44 & 0.23 &  &  &  &  &  &  &  \\
Ethane (C$_{2}$H$_{6}$) &  &  &  & 0.41 & 0.13 &  &  &  &  &  &  &  \\
Methanol (CH$_{3}$OH) &  &  &  & 1.01 &  &  &  &  &  &  &  &  \\
Formaldehyde (HCHO) &  &  &  & 0.6 &  &  &  &  &  &  &  &  \\
Acetic Acid (CH$_{3}$COOH) &  &  &  & 2.62 &  &  &  &  &  &  &  &  \\
Formic Acid (HCOOH) &  &  &  & 0.063 &  &  &  &  &  &  &  &  \\
Ammonia (NH$_{3}$) &  &  &  & 0.79 &  &  &  &  &  &  &  & 0.0009 \\
Nitrogen Oxides (NOx) &  &  &  & 1.41 &  &  &  &  & 2 &  &  &  \\
Nitrous Oxide (N$_{2}$O) & 0.118 & 0.02235 & 0.87 & 0.24 &  &  &  &  &  &  &  &  \\
NMOC &  &  &  & 11.1 &  & 6 & 10 &  &  &  &  &  \\
Total PAH &  &  &  &  &  &  &  & 0.000455 & 0.0065 &  &  &  \\
Acetaldehyde &  &  &  &  &  &  &  &  & 0.545 &  &  &  \\
VOC &  &  &  &  &  &  &  &  &  & 14 & 4.5 &  \\ \bottomrule
\end{tabular}
} %end resize
\end{table}     
\end{sidewaystable}

It is interesting to note that the IPCC factors for CO$_{2}$ are physically impossible to achieve as the maximum amount of CO$_{2}$ that could be produced from 1 kg of pure carbon is only 3,667 g, but most charcoal only has a maximum total organic carbon content of 70\%, making a realistic limit of 2,567 as shown in the Bhattacharya (2002) column.  This project discovered this issue, which was acknowledged by the IPCC in communication to the authors. 

Factors from burning mixed biomass, leaves, and weeds from AP-42 were included because they represent analogous processes of burning tobacco.  Their factors for CO and CH$_{4}$ are within reported ranges from other sources.  The AP-42 factor of NH$_{3}$ from cigarettes is several orders of magnitude smaller than the value reported by Akagi (2011). A compilation of factors based on Table \ref{tb1:initialfactors} is shown in Table \ref{tb2:emissionfactors}. The factors are divided into minimum, maximum and expected values that are used to define Triangle distributions. Triangle distributions are recommended forms when the underlying distribution is not known, no or little data is available to calculate parameters, and the Central Limit Theorem cannot by invoked \citep{Firestone1997, Lipton1995, Salling2008}. 

\begin{table}[H]
\centering
\caption{Emission factors for $nargyla$ pipe emissions in $g/kg$}
\label{tb2:emissionfactors}
\begin{tabular}{@{}lcccc@{}}
\toprule
\textbf{Pollutant} & \textbf{Min} & \textbf{Expected} & \textbf{Max} & \textbf{Source} \\ \midrule
CO$_{2}$ & 2,155 & 2,385 & 2,567 & Bhattacharya (2002) \\
CO & 35 & 189 & 198 & Bhattacharya (2002) \\
CH$_{4}$ & 5.29 & 6.7 & 7.8 & Bhattacharya (2002) \\
N$_{2}$O & 0.118 & 0.24 & 0.87 & Paciornick(2006) \\
NMOC & 6 & 10 & 11 & Bhattacharya (2002) \\
PAH & 0.0001 & 0.00045 & 0.0065 & Bhattacharya (2002)/USEPA (1995) \\
NOx & 0.5 & 1.41 & 2 & Paciornick(2006) \\
NH$_{3}$ & 0.0009 & 0.395 & 0.79 & Bhattacharya (2002)/USEPA (1995) \\ \bottomrule
\end{tabular}
\end{table}

\subsection{Calculating annual emissions}

The variation of results associated with estimating annual emissions of a source given the multiple independent variables that go into calculating the output can be handled using MCA.  By randomly drawing values for each variable from an underlying distribution within the expected range of possible results, the MCA generates results that represent a range of possible outcomes given the input set \citep{Johnson2011}.  For annual emissions, the primary equation for individual emissions for a source is given as
%
\begin{equation}
\label{eq1}
Q_{i} = EF_{i} * AC_{X}
\end{equation}
%
\noindent
where $Q$ is the emissions of pollutant $i$, $EF$ is the individual emission factor for the pollutant resulting from process $X$, and $AC$ is the amount of input feedstock used by process $X$ over the reporting period.  In this case, there is only one process and the reporting period is one year.  With the $EF$’s for each chemical in Table \ref{tb2:emissionfactors} given in grams of emissions per kilogram of material consumed, the value for AC in this process is measured in total kilograms of material consumed in combustion and calculated as
%
\begin{equation}
\label{eq2}
AC = AC_{charcoal} + ( \lambda * AC_{sheesha} )
\end{equation}
%
\noindent
where $AC_{charcoal}$ is the total mass of charcoal consumed during combustion and $AC_{sheesha}$ is the total mass of $sheesha$ consumed during combustion.  Because the $sheesha$ is not fully burned during the smoking process, only a fraction of the total amount is assumed to contribute to the total generated emissions.  A scaling term, lambda, $\lambda$, is introduced to account for this fractional portion. 

Annual consumption of charcoal and $sheesha$ was estimated by sampling individual restaurants and caf\'es that serve $nargyla$ pipes to clients.  Our work requested from managers to determine how much charcoal and tobacco were used daily, weekly, or monthly.  Response ranged from 15-35 kg of charcoal per day (100-140 kg/week) and 1-10 kg of $sheesha$ tobacco (all flavors) per day (7-70 kg/week).  Because the number of smokers at each caf\'e can be assumed to be randomly and independently drawn from their underlying PDFs, the CLT can be used to simplify the calculation process resulting in Normal distributions that represent the sum of all charcoal and $sheesha$ used on any given day.  The normal distribution has mean ($\mu$) and standard deviation ($\sigma$) input parameters defined as
%
\begin{equation}
\label{eq3}
\mu_{CLT}= \mu*N
\end{equation}
%
\noindent
and 
%
\begin{equation}
\label{eq4}
\sigma_{CLT}= \sigma*\sqrt{N}
\end{equation}
%
\noindent
where $N$ is the number of samples (in this case the number of caf\'es and restaurants), $\mu$ is the sample mean, $\sigma$ is the sample standard deviation, $\mu_{CLT}$ is the CLT mean, and $\sigma_{CLT}$ is the CLT standard deviation \citep{Ott1981}. 

The number of establishments serving $nargyla$ can be assumed to be constant with a few new shops opening or old shops closing.  This work sampled caf\'es and restaurants and included well-known shops to assure a better representation of the total population of $sheesha$ establishments.   Most of the caf\'es and restaurants that serve $nargyla$ in Kuwait are located in the Salmiya and Hawalli areas, with other concentrations in Jeleeb Al Shuyoukh, Abu Fatira, Fahaheel, and Jahra.  Most hotels also have a caf\'e that serves $nargyla$, as well as food malls such as the Arbilla in Al-Bidda.  Figure \ref{figng2:cafes} shows clusters of caf\'es in general geographic areas.  For this study, 86 establishments are included that are assumed to use the same average amount on an annual basis.  Table \ref{tb3:feedstock} shows the calculated annual averages and standard deviations of charcoal and $sheesha$ utilization as well as the total consumed amount per Equation \ref{eq2} for an individual establishment.  For modeling purposes, mean annual consumption and standard deviation were calculated by multiplying the weekly averages by 52 weeks and applying the CLT factors from Equations \ref{eq3} and \ref{eq4} as shown in Equations \ref{eq5} and \ref{eq6} respectively.  The values of $N$ in Equations \ref{eq5} and \ref{eq6}4 represent the active establishments used in the study.  In this case, $N$ = 86.
%
\begin{equation}
\label{eq5}
\mu_{Annual}= \mu*52*N
\end{equation}
%
\noindent
and 
%
\begin{equation}
\label{eq6}
\sigma_{Annual}= \sigma*52*\sqrt{N}
\end{equation}
%
\noindent

%
\begin{table}[H]
\centering
\caption[Feedstock inputs for individual establishments and total annual consumption]{Feedstock inputs (in kgs) for individual establishments and total annual consumption}
\label{tb3:feedstock}
\resizebox{\columnwidth}{!}{%
\begin{tabular}{@{}lcccccc@{}}
\toprule
 & \multicolumn{4}{c}{\textbf{Individual establishment consumption}} & \multicolumn{2}{c}{\textbf{Total Consumption}} \\ 
\textbf & \textbf{Weekly} & \textbf{Weekly} & \textbf{Weekly} & \textbf{Weekly} & \textbf{Annual} & \textbf{Annual total} \\ 
\textbf{Feedstock} & \textbf{Min} & \textbf{Max} & \textbf{Mean} & \textbf{Std Dev} & \textbf{total mean} & \textbf{Std Dev} \\ \midrule
Charcoal & 100 & 140 & 120 & 20 & 536,640 & 9,644 \\
$sheesha$ & 7 & 70 & 38.5 & 31.5 & 172,172 & 15,190 \\ \bottomrule
\end{tabular}
}%end resize
\end{table}
%

A $\lambda = 0.2$ was assumed to account for the $sheesha$ tobacco burned during the smoking process based on study evaluations.  Lambda was held constant for this paper but could vary based on smoking habits and pipe preparation.  The annual total material consumed (AC) mean ($\mu$) and standard deviation ($\sigma$) for charcoal and $sheesha$ were calculated using MCA.  The variables in Equation \ref{eq2} were replaced with Normal distributions, $N(\mu,\sigma)$,  using the means and standard deviations of the total annual consumptions from Table \ref{tb3:feedstock}.  The evaluated equation            
%
\begin{equation}
\label{eq7}
AC = N(536640.0,  9644.0) + \lambda*N(172172.0, 15190.0)
\end{equation}
%
\noindent
provided a total average annual consumption mean of 571,071 kg and a standard deviation of 10,110 kg after 50,000 iterations using @Risk 7.5.1 with Latin Hypercube sampling (\url{www.palisades.com}).  The annual mass consumption was represented using a Normal distribution with the parameters $N(571071, 10110)$.

This study did not include $nargyla$ usage at home nor did it include charcoal consumption used by restaurants strictly for grilling foods.  Additional emissions are generated during grilling due to the combustion of animal fats \citep{McDonald2003}.  Grilling emissions should be incorporated in an emissions inventory but were not considered suitable for this study.

%
\begin{figure}[H]
\includegraphics[width=\textwidth,height=\textheight,keepaspectratio]{images/ng2.png} 
\caption{Location clusters of caf\'es with $nargyla$.}
\label{figng2:cafes}
\end{figure}

\section{Results}

Using the general form of annual emissions in Equation \ref{eq1}, the Emissions Factor PDFs from Table \ref{tb2:emissionfactors} were converted into Triangle distributions, and the parameters for total annual combustion in Equation 5 were used to form a Normal distribution to represent the total input feedstock.  An MCA was performed for 50,000 realizations using @RISK 7.5.1 with Latin Hypercube sampling. The number of iterations was selected to ensure full coverage and is significantly more than the minimum amount needed to cover over 99\% of all possible scenarios.  If system resources and complexity required reduced iterations, methods exist to more precisely estimate the minimum number required for a desired confidence interval \citep{Bukaci2016}.

Results were divided by 106 to get units of metric tonnes, or 103 to get kilograms if the total was less than 1 tonne per year.  The average annual emissions for individual pollutants are shown in Table \ref{tb4:results} with Triangle distributions of the EFs and results shown in Figure \ref{figng3:pdfs}. 

%
\begin{table}[H]
\centering
\caption{MCA Results for Annual Emissions for $nargyla$ smoking in Kuwait.}
\label{tb4:results}
\begin{tabular}{@{}cccc@{}}
\toprule
 & \textbf{Pollutant} & \textbf{Annual Mean} & \textbf{Annual Std} \\ \midrule
\multirow{4}{*}{\textbf{tonnes}} & CO$_{2}$ & 1,352.88 & 53.32 \\
 & CO & 80.33 & 21.40 \\
 & NMOC & 5.14 & 0.62 \\
 & CH$_{4}$ & 3.77 & 0.30 \\ \midrule
\multirow{4}{*}{\textbf{kg}} & NOx & 744.30 & 176.64 \\
 & N$_{2}$O & 233.76 & 94.17 \\
 & NH$_{3}$ & 225.75 & 92.07 \\
 & PAH & 1.34 & 0.84 \\ \bottomrule 
\end{tabular}
\end{table}
% 
 
     
%
\begin{figure}
%\includegraphics[width=\linewidth,height=22.1cm,keepaspectratio]{images/ng3.png}
\includegraphics[width=\textwidth,height=\textheight,keepaspectratio]{images/ng3.png}  
\caption[Results of Emission Factor PDFs and Annual Emissions]{Results of Emission Factor PDFs and Annual Emissions based on Triangle distributions of Emission Factors.}
\label{figng3:pdfs}
\end{figure}
%

\subsection{Sensitivity Analysis or Area Source Emissions}
While the annual mass consumption could assume a Normal distribution due to the Central Limit Theorem, the EFs were arbitrarily assumed to have Triangle distributions.  To evaluate the selection of distribution, different distributions were applied to the MCA and compared to the Triangle distribution results.  Additional distributions included the Uniform, Normal, and Pert. Of the four distributions compared, all used finite range parameters based on the Min, Max, and Expected values of Table \ref{tb2:emissionfactors}.  For the Normal distribution, the mean, $\mu$, was assumed to be the average of the Min and Max, while the standard deviation, $\sigma$, assumed that the Min/Max were 3$\sigma$ from the mean and was calculated as
%
\begin{equation}
\label{eq8}
\sigma = (\mu – Min) / 3
\end{equation}
%
The parameters used for the different EF distributions are shown in Table \ref{tb5:sensitivity}.
%
\begin{table}[H]
\centering
\caption{Sensitivity Analysis Parameters for EF Distributions g/kg}
\label{tb5:sensitivity}
\begin{tabular}{@{}lccccc@{}}
\toprule
\textbf{Pollutant} & \textbf{Min} & \textbf{Expected} & \textbf{Max} & \textbf{$\mu$} & \textbf{$\sigma$} \\ \midrule
CO$_{2}$ & 2155 & 2385 & 2567 & 2361 & 68.7 \\
CO & 35 & 189 & 198 & 116.5 & 27.2 \\
NMOC & 6 & 10 & 11 & 8.5 & 0.8 \\
CH$_{4}$ & 5.29 & 6.7 & 7.8 & 6.545 & 0.4 \\
NOx & 0.5 & 1.41 & 2 & 1.25 & 0.25 \\
N$_{2}$O & 0.118 & 0.24 & 0.87 & 0.494 & 0.13 \\
NH$_{3}$ & 0.0009 & 0.395 & 0.79 & 0.39545 & 0.13 \\
PAH & 0.0001 & 0.00045 & 0.0065 & 0.0033 & 0.001 \\ \bottomrule
\end{tabular}
\end{table}
%
Comparing the means of each distribution shows that changes in the average vary slightly among the different forms with most variation, represented by the sample standard deviation (s) being N$_{2}$O in Table \ref{tb6:compare}.  The Triangle and Pert distributions provide the most conservative average values assuming central tendencies apply to the distribution.
%
\begin{table}[H]
\centering
\caption{Comparison of annual total emission means}
\label{tb6:compare}
\resizebox{\columnwidth}{!}{%
\begin{tabular}{@{}cccccccccc@{}}
\toprule
 & \multicolumn{5}{c}{\textbf{Distributions}} & \multicolumn{4}{c}{\textbf{Statistics}} \\ \midrule
\textbf{Units} & \textbf{Pollutant} & \textbf{Triangle} & \textbf{Uniform} & \textbf{Normal} & \textbf{Pert} & \textbf{$\bar\{x\}$} & \textbf{S} & \textbf{Lower 95\% CI} & \textbf{Upper 95\% CI} \\ \midrule
\multirow{4}{*}{\textbf{tonnes}} & CO$_{2}$ & 1352.9 & 1348.3 & 1348.3 & 1357.4 & 1351.7 & 4.37 & 1,343.2 & 1,360.3 \\
 & CO & 80.3 & 66.5 & 66.5 & 94.1 & 76.9 & 13.2 & 51.0 & 102.8 \\
 & NMOC & 5.1 & 4.9 & 4.9 & 5.4 & 5.1 & 0.3 & 4.5 & 5.6 \\
 & CH$_{4}$ & 3.8 & 3.7 & 3.7 & 3.8 & 3.8 & 0.03 & 3.7 & 3.8 \\ \midrule
\multirow{4}{*}{\textbf{kg}} & NOx & 744.3 & 713.8 & 713.8 & 774.8 & 736.7 & 29.2 & 679.5 & 793.8 \\
 & N$_{2}$O & 233.8 & 282.1 & 282.1 & 185.4 & 245.8 & 46.3 & 155.1 & 336.6 \\
 & NH$_{3}$ & 225.7 & 225.8 & 225.8 & 225.7 & 225.8 & 0.1 & 225.6 & 225.9 \\
 & PAH & 1.3 & 1.9 & 1.9 & 0.8 & 1.5 & 0.5 & 0.5 & 2.5 \\ \bottomrule
\end{tabular}
} %end resize
\end{table}
%
A similar review of the standard deviations of the different distributions shows that the NOx, N$_{2}$O, and NH$_{3}$ have large variances relative to their means as seen in Table \ref{tb7:stndev}.  In terms of distributions, the Uniform distribution offers the most variance with the Triangle distribution coming next. In cases like this where ranges of values have finite limits, using a Normal distribution or any of the t-distribution families is not recommended as possible MCA values could be out of physical range.
%
\begin{table}[H]
\centering
\caption{Comparison of annual total emission standard deviations.}
\label{tb7:stndev}
\resizebox{\columnwidth}{!}{%
\begin{tabular}{@{}cccccccccc@{}}
\toprule
 & \multicolumn{5}{c}{\textbf{Distributions}} & \multicolumn{4}{c}{\textbf{Statistics}} \\ \midrule
\textbf{Units} & \textbf{Pollutant} & \textbf{Triangle} & \textbf{Uniform} & \textbf{Normal} & \textbf{Pert} & \textbf{$\bar{x}$} & \textbf{S} & \textbf{Lower 95\% CI} & \textbf{Upper 95\% CI} \\ \midrule
\multirow{4}{*}{\textbf{tonnes}} & CO$_{2}$ & 53.8 & 71.9 & 45.8 & 50.4 & 55.5 & 11.43 & 33.1 & 77.9 \\
 & CO & 21.4 & 26.9 & 15.6 & 14.3 & 19.5 & 5.8 & 8.1 & 30.9 \\
 & NMOC & 0.6 & 0.8 & 0.5 & 0.5 & 0.6 & 0.2 & 0.3 & 0.9 \\
 & CH$_{4}$ & 0.3 & 0.4 & 0.2 & 0.3 & 0.3 & 0.08 & 0.2 & 0.5 \\ \midrule
\multirow{4}{*}{\textbf{kg}} & NOx & 176.7 & 247.6 & 143.4 & 160.8 & 182.1 & 45.7 & 92.5 & 271.8 \\
 & N$_{2}$O & 94.2 & 124.1 & 71.6 & 72.6 & 90.6 & 24.7 & 42.3 & 138.9 \\
 & NH$_{3}$ & 92.1 & 130.2 & 75.5 & 85.3 & 95.8 & 23.9 & 48.9 & 142.7 \\
 & PAH & 0.840 & 1.055 & 0.572 & 0.556 & 0.8 & 0.2 & 0.3 & 1.2 \\ \bottomrule
\end{tabular}
} %end resize
\end{table}

While the form of the EF distribution influences annual emission distributions, the influence is not significant. Ninety-five percent (95\%) Confidence Intervals (CI) of the distribution means are shown in Tables \ref{tb6:compare}, with similar CI’s for the sample standard deviations in Table \ref{tb7:stndev}. In all case, the distribution statistics are within the bounds and therefore can accept a null hypothesis that distribution means are not significantly different and that distribution variances are not statistically significant ($p<0.05$).  Tables \ref{tb6:compare} and \ref{tb7:stndev} show that using a Triangular distribution provides a conservative value if the underlying distribution is not known and few data points are available to describe the sample population.  

\subsection{Comparison of Annual Results to National Emissions Inventory}
Applying the resulting means generated from the assigned PDFs we can compare the estimated emissions from smoking $nargyla$ to the total emissions reported by Kuwait in its first national communication under the United Nations Framework Convention on Climate Change (UNFCCC) \citep{AlMudhhi2012}.  The report baselines Kuwait’s Greenhouse Gas (GHG) emissions for 1996 baseline year.  Table \ref{tb8:comparison} shows that the estimated emissions from smoking in 2016 are significant compared to the entire reported emissions of 1996. 

%
\begin{table}[H]
\centering
\caption{Comparison of 2016 $nargyla$ emissions to 1996 total emissions in Gg}
\label{tb8:comparison}
\begin{tabular}{@{}cccc@{}}
\toprule
 & \textbf{Annual 2016} & \textbf{1996} & \textbf{Percent of} \\
\textbf{Pollutant} & \textbf{$nargyla$ emissions} & \textbf{total emissions} & \textbf{1996 emissions} \\ \midrule
CO$_{2}$ & 1,352.9 & 29,502 & 4.6\% \\
CO & 80.3 & 544 & 14.8\% \\
NMOC & 5.1 & 522 & 1.0\% \\
CH$_{4}$ & 3.8 & 129.19 & 2.9\% \\
NOx & 0.74 & 113 & 0.7\% \\
N$_{2}$O & 0.23 & 0.44 & 53.1\% \\ \bottomrule
\end{tabular}
\end{table}

This comparison may be unfair as the 1996 emissions were probably lower than emissions released in more recent years.  For one reason, the population in 1995 was reported as 1,575,570 (total Kuwaiti and non-Kuwaiti) while in 2016, the total population was reported as 4,132,415 \citep{CSB2017}.  If we assume that the 1996 $nargyla$ emissions would be a 38\% linear approximation of the 2016 emissions (the percentage of the 1996 population to the 2016 population – we can assume the same percentage of smokers in both populations), the adjusted comparison in Table \ref{tb9:adjusted} still shows that $nargyla$ smoking is large relative to other combustion processes, especially N$_{2}$O.
%
\begin{table}[H]
\centering
\caption{Comparison of $nargyla$ emissions in Gg adjusted to 1996 population.}
\label{tb9:adjusted}
\begin{tabular}{@{}cccc@{}}
\toprule
 & \textbf{Annual 2016} & \textbf{1996} & \textbf{Percent of} \\
\textbf{Pollutant} & \textbf{$nargyla$ emissions} & \textbf{total emissions} & \textbf{1996 emissions} \\ \midrule
CO$_{2}$ & 514.1 & 29,502 & 1.7\% \\
CO & 30.5 & 544 & 5.6\% \\
NMOC & 2.0 & 522 & 0.4\% \\
CH$_{4}$ & 1.4 & 129.19 & 1.1\% \\
NOx & 0.28 & 113 & 0.3\% \\
N$_{2}$O & 0.09 & 0.44 & 20.2\% \\ \bottomrule
\end{tabular}
\end{table}

\section{Conclusions}

Recreational smoking using $nargyla$ pipes is a popular pastime for many people in the Middle East and is growing in other parts of the world.  The major emissions are generated from the combustion of charcoal with a fractional contribution from the flavored $sheesha$ tobacco.  This study looked at the gaseous emissions associated with smoking $nargyla$ pipes that could contribute to the overall air pollution emissions inventory.  Emission rates of different gases were collected from previous research and published emission factors.  The emission factors were converted into triangle PDFs to account for variable rates.  Assuming that the many restaurants and caf\'es are serving $nargyla$ use the similar annual average amounts of charcoal and $sheesha$ tobacco, the Central Limit Theorem could be used to calculate annual feedstock consumption for the entire country.  A Monte Carlo Analysis was used to calculate the total emissions of individual pollutants.  Resulting values showed that GHG emissions were large, while carcinogenic PAH emissions were low overall.  Evaluating the risk associated with exposure to the individual emissions was not part of the study, but is covered in other research \citep{Fromme2009, Moon2015, Mulla2015}.

This investigation demonstrated that selecting the underlying distribution for the emissions factors was important and that Triangle and Pert distributions provide more conservative results and can, therefore, be used when the data set is unknown or limited.  The final $nargyla$ emissions results were significant when compared to the reported total emissions of 1996 reported in Kuwait’s initial correspondence report for the UNFCCC.  Even when the results were adjusted to the reduced population of 1996 (38\% of the current population in 2016), the results are still large compared to the total reported amount. While a conservative estimation was used for the emission factors during the IPCC report preparation, the actual amount of $nargyla$ smoked was not accounted for. Gaps in the inventory include emissions from home use and caf\'es that might not have been accounted for.

A more realistic assumption is that the 1996 baseline did not capture all the emissions. Using a relatively easy emission source such as $nargyla$ smoking can provide a rapid check of inventory totals to confirm whether estimation processes make sense and are generating expected results. It also highlights where the attention of different pollutants is placed.  Calculations of N$_{2}$O in the 1996 baseline need additional review, especially if $nargyla$ smoking represents 20\% of all reported N$_{2}$O emissions.  The Greenhouse Warming Potential (GWP) of N$_{2}$O is 289 times greater than CO$_{2}$ over 100 years (compared to 24 times for CH$_{4}$)  and is, therefore, an important GHG \citep{IPCC2007}.

While the main purpose of this study was to show how limited data can still be used to generate macro-scale results for purposed of national inventories, the study was nonetheless limited to the accuracies of the emission factors employed.  The published values may not have been accurate representations of the materials used, either overestimating or underestimating the emission rates. Our final figures, therefore, can only represent a range of possible solutions, rather than a definitive answer.  What the results of the study provide is an order of magnitude to compare other results and a basis to prepare more accurate studies.

%---------------------------------------------------------------
%------------------------End of Chapter----------------------
\bigskip
\begin{center}
END OF CHAPTER
\end{center}