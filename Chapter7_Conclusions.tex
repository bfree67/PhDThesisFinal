\chapter{Conclusions and Recommendations}
This chapter provides a summary of the research findings, limitations, and recommendations for future work.

\section{Summary of contributions}

The research presented in the previous chapters addressed several key areas required to develop and maintain a comprehensive air management program effectively. The major novel academic contributions of this thesis are as follows:

\begin{itemize}
\item {Using synthetic sources and CALPUFF air dispersion modeling to determine common air mixing to identify manageable air quality zones.}
\item {Development of a stochastic model to compare exceedance criteria for zones with multiple air monitoring stations using a risk-based approach}
\item{Development of a stochastic model to quantify small and dispersed area sources for inclusion in national emissions inventories.}
\item{Use of a UAS to capture traffic imagery at signalized intersections and convert into 3D models to evaluate stacking patterns and fleet composition as inputs into mobile emission models}.
\item{Use of deep learning to predict near-term air pollution concentrations using only locally collected historical data to train and test a recurrent neural network with long short term memory.}
\end{itemize}


\section{Limitations and Recommendations}

The following observations are presented within each task that identifies key limitations within the presented research and recommendations for follow-on studies.

\noindent
\subsection*{Identifying air zones}

\noindent
\textbf{Observation:}\\
The two areas investigated, Kuwait and Qatar, share common topography, land use features, and climate. Other coastal communities with more diverse topography and climates may have different dispersion profiles and be subject to different LSB effects. The methodology presented in this task should be applied to other coastal areas at different latitudes and topography.\\
\noindent\\
\textbf{Observation:}\\
Only SO$_{2}$ was used as a dispersant gas in the virtual sources. Other pollutants such as CO and NO$_{2}$ were not evaluated. Different emission gases should be used in modeling and compared to the dispersion of SO$_{2}$. Substituting different pollutants may be more appropriate for an area that has different source types.\\
\noindent\\
\textbf{Observation:}\\
Modeling outputs from WRF and MM5 prognostic meteorological data were compared, but the prognostic results were never compared to actual observations. Calibration of the models may be necessary to bring the results in line with surface observations available from local weather stations. No study was available that compares prognostic weather data to actuals. This study would be applicable to researchers and consultants using dispersion models in Kuwait.

\subsection*{Exposure risk from zone classification}
\noindent
\textbf{Observation:}\\
A composite subpopulation group was used to evaluate adult body weights instead of assessing individual group separately. These groups can be further broken down by gender, age, and work type. Some jobs require more time outside or long commutes from the place of residence to the place of work. Specific subgroups should be evaluated to determine their susceptibility to exposure risk.\\

\noindent
\textbf{Observation:}\\
Only O$_{3}$ and NO$_{2}$ were considered in the research without consideration of their fate or possible transformations. Additionally, other pollutants were not considered that have different time weighting standards. 

\noindent
\textbf{Observation:}\\
The PDFs assigned to the Attainment and Exceedance distributions were based on ranking curve-fitting statistics. These statistics are themselves summaries of the distributions and may not reflect the underlying distribution of the sample population.

\noindent
\textbf{Observation:}\\
The selection of Attainment and Exceedance samples assumed independence for each sample, however, in reality, the preceding measurements influence the current and future measurement in continuous time series systems. This property is exploited in the research to forecast concentrations.

\subsection*{Estimating annual emissions from distributed area sources}
\noindent
\textbf{Observation:}\\
A limited number of facilities were considered for evaluation. An exhaustive audit was not conducted, nor were personal sources considered. A more thorough audit of facilities should be conducted as well as surveys of managers to get individual consumption rates of charcoal and tobacco. \\

\noindent\\
\textbf{Observation:}\\
Emission factors from the literature were used to estimate emissions. The accuracy of these factors is always questionable, and errors can be assumed. Actual measurements of emissions from different charcoal and $sheesha$ tobacco should be conducted to characterize better the factors used.\\

\noindent\\
\textbf{Observation:}\\
Usage was only calculated for public establishments and did not account for private and residential use. This could be a significant amount in regards to under-reporting emissions.\\

\subsection*{Estimating traffic density}
\noindent
\textbf{Observation:}\\
Flights and data collection were taken only at two different locations at two different times. More data should be collected from different locations, different times of day,  different days of the week, and different weeks of the year to account for possible seasonal effects. \\

\noindent\\
\textbf{Observation:}\\
Measurements were only taken at 3 lane signalized intersections. Other traffic control features were not measured, such as 1 and 2 lane signalized intersections, traffic circles, merge lanes, entrance and exit ramps, and 4-way intersections.  Stacking distances and fleet compositions should be evaluated at these features, especially it some features are common, such as 1 lane signalized intersections.\\

\noindent\\
\textbf{Observation:}\\
The results are only applicable to the local area where the data was collected unless other datasets show similar features. Driver profiles such as age, background, and experience may impact stacking distances, while different fleet compositions will impact overall vehicle types.

\subsection*{Forecasting environmental time series using deep learning}
\noindent
\textbf{Observation:}\\
A deep learning system was trained on the data from one AMS only. Most stations are part of a network that can provide additional predictive support due to weather patterns and nearest neighbor influence. Evaluation of results using features from different AMS's should be compared to the results of a single AMS.\\

\noindent\\
\textbf{Observation:}\\
Only 8 hr ave O$_{3}$ was used for training and evaluation. Other measured pollutants were not evaluated. Primary pollutants such as CO or SO$_{2}$  may have different feature influences and require different features for accurate predictions.\\

\noindent\\
\textbf{Observation:}\\
The deep learning network was trained on 1 hr intervals. Some pollutants such as PM$_{10}$ and PM$_{2.5}$ are measured on 24 hr scales. Different input intervals should be trained and tested.

\noindent\\
\textbf{Observation:}\\
Deep learning techniques are black boxes that do not allow a fundamental understanding of the processes used to generate the output. They are powerful methods when the ends justify the means, but may not be useful for all applications.

\bigskip
%---------------------------------------------------------------
%------------------------End of Chapter----------------------

\begin{center}
END OF CHAPTER
\end{center}